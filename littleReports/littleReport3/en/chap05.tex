\chapter{The role of geodesics}
\textit{\today\newline
Jan Střeleček\newline}



\section{Minimizing the energy variance}
From \cite{Bukov2019}.
About transports using \emph{fast forward} Hamiltonian means the system is driven to the target state in some fixed amount of time. The transport is done on the ground state manifold $\M$.

\conjecture{
    For any fast forward Hamiltonian $\HH(\lambda(t))$ driven along one dimentional path $\lambda(t): \R\mapsto \R$ using time $t$ as parametrization, the energy fluctuations $\delta E^2$, averaged along the path, are larger than the geodesic length $l_\lambda$
    \begin{equation}
        \int\limits_0^T\sqrt{\delta E^2(t)}\d t \eqqcolon l_t\geq l_\lambda \int\limits_{\lambda_i}^{\lambda_f} \sqrt{g_{\lambda\lambda}} \d \lambda=\int_0^T \sqrt{g_{\lambda\lambda}}\frac{\d \lambda}{\d t}\d t.
    \end{equation}
    The length $l_\llambda$ is defined in control space (with metric tensor $g_{\lambda\lambda}$) and is generally larger than the distance between wave functions, i.e. the absolute geodesic (defined with $G_{\mu\nu}$). From its definition, we can see that it corresponds to the metric tensor as we use it.
    
    
    
    The energy variance is 
    \begin{equation}
        \delta E^2= \braket{o(t)|\HH(t)^2|o(t)}-\braket{o(t)|\HH(t)|o(t)}^2=\braket{\partial_t (t)|\partial_t o(t)})_c=G_{tt}
    \end{equation}    
    and the Metric tensor in control space is defined as
    \begin{equation}
        g_{\lambda\lambda}\coloneqq \braket{\partial_\lambda o(t)|\partial_\lambda o(t)})_c
    \end{equation}
}


\begin{proof}
    \begin{equation}
        \delta E^2\equiv \braket{o(t)|\HH(t)^2|o(t)}_c=\dot\lambda^2 G_{\lambda\lambda}+\O(\dot\lambda^4),
    \end{equation}
    where $\O(\dot\lambda^4)$ needs to be positive for any real-valued Hamiltonian. This comes from the fact, that it has instantaneous time-reversal symmetry.
\end{proof}

The conjecture only applies to unit fidelity protocols and can be extended to an arbitrary dimensional path.



% \section{Solving the \Schrodinger equation}
% \cite{DeGrandi2010}



\section{APT}
From \cite{Rigolin2008}.

The power series will be derived using a small parameter $v=1/T$.

Starting again with
\begin{equation}
    \ket{\Psi(s)}=\sum_{p=0}^\infty v^p \ket{\Psi^{(p)}(s)},
    \label{eq:mainSeries}
\end{equation}

for 
\begin{equation}
    \ket{\Psi(s)}=\sum_{n=0} e^{-\frac{i}{v} \textcolor{purple}{\omega_n(s)}}e^{i\textcolor{violet}{\gamma_n(s)}}\textcolor{blue}{b_n^{(p)}(s)}\ket{n(s)}.
\end{equation}
Here the
\begin{align}
    \textcolor{purple}{\omega_n(s)} &\coloneqq \frac{1}{\hbar}\int_0^s E_n(s')\d s'\\
    \textcolor{violet}{\gamma_n(s)} &\coloneqq i\int_0^s \braket{n(s')|\frac{\d}{\d s'}n(s')}\d s' \equiv i\int_0^s M_{nn}(s')\d s'
    \label{eq:gammadef}
\end{align}
are so-called dynamical resp. Berry (geometric) phase and $\ket{n(s)}$ are solution to
\begin{equation}
    \HH(s)\ket{n(s)}=E_n(s) \ket{n(s)}.
\end{equation}
The problem again lies in determining $\textcolor{blue}{b_n^{(p)}(s)}$, which will be done iteratively. Because it is dependent on its relative geometric and dynamical phases to other energy levels, lets write it as a series
\begin{equation}
    \textcolor{blue}{b_n^{(p)}(s)}=\sum_{m=0} e^{\frac{i}{v}\textcolor{purple}{\omega_{nm}(s)}}e^{-i\gamma_{nm}(s)}\textcolor{blue}{b_{nm}^{(p)}(s)},
\end{equation}
where $\textcolor{purple}{\omega_{nm}} \coloneqq \textcolor{purple}{\omega_m}-\textcolor{purple}{\omega_n}$, $\textcolor{violet}{\gamma_{nm}} \coloneqq \textcolor{violet}{\gamma_m}-\textcolor{violet}{\gamma_n}$. The new coefficients $\textcolor{blue}{b_{mn}^{(p)}(s)}$ now \textbf{depend only on the metric structure around the path} and therefore one may perform adiabatic expansion from any point on manifold.

Inserting all to original series \ref{eq:mainSeries}, we get
\begin{equation}
    \ket{\Psi(s)}=\sum_{n,m=0}\sum_{p=0}^\infty v^p e^{-\frac{i}{v}\textcolor{purple}{\omega_m(s)}}e^{i\textcolor{violet}{\gamma_m(s)}}\textcolor{blue}{b_{nm}^{(p)}(s)}\ket{n(s)}.
    \label{eq:solve0}
\end{equation}

Because the initial state is eigenstate, we get initial conditions $b_{nm}^{(0)}(s)=0$. In addition, one can rewrite equation \ref{eq:solve0} to the iteratively solvable form
\begin{equation}
    \frac{i}{\hbar}\Delta_{nm}(s)\textcolor{blue}{b_{nm}^{(p+1)}(s)}+\textcolor{blue}{\dot b_{nm}^{(p)}(s)}+W_{nm}(s) \textcolor{blue}{b_{nm}^{(p)}(s)}+\sum_{k=0,k\neq n}M_{nk}(s)\textcolor{blue}{b_{km}^{(p)}(s)}=0,
    \label{eq:bSolution}
\end{equation}
for $W_{nm}(s)\coloneqq M_{nn}(s)-M_{mm}(s)$, where $M_{mn}$ is defined in Eq. \ref{eq:gammadef}. We can see that $b_{mn}^{(p)}$, as a solution to Eq. \ref{eq:bSolution}, only depends on difference between energy levels, eigenstates during the path and their directional derivatives. All of those are easily obtained, once the driving path is prescribed.

    