\chapter{The role of geodesics}


\section{Minimizing the energy variance}
From \cite{Bukov2019}.
About transports using \emph{fast forward} Hamiltonian means the system is driven to the target state in some fixed amount of time. The transport is done on the ground state manifold $\M$.

\conjecture{
    For any fast forward Hamiltonian $\HH(\lambda(t))$ driven along one dimentional path $\lambda(t): \R\mapsto \R$ using time $t$ as parametrization, the energy fluctuations $\delta E^2$, averaged along the path, are larger than the geodesic length $l_\lambda$
    \begin{equation}
        \int\limits_0^T\sqrt{\delta E^2(t)}\d t \eqqcolon l_t\geq l_\lambda \int\limits_{\lambda_i}^{\lambda_f} \sqrt{g_{\lambda\lambda}} \d \lambda=\int_0^T \sqrt{g_{\lambda\lambda}}\frac{\d \lambda}{\d t}\d t.
    \end{equation}
    The length $l_\llambda$ is defined in control space (with metric tensor $g_{\lambda\lambda}$) and is generally larger than the distance between wave functions, i.e. the absolute geodesic (defined with $G_{\mu\nu}$). From its definition, we can see that it corresponds to the metric tensor as we use it.
    
    
    
    The energy variance is 
    \begin{equation}
        \delta E^2= \braket{o(t)|\HH(t)^2|o(t)}-\braket{o(t)|\HH(t)|o(t)}^2=\braket{\partial_t (t)|\partial_t o(t)})_c=G_{tt}
    \end{equation}    
    and the Metric tensor in control space is defined as
    \begin{equation}
        g_{\lambda\lambda}\coloneqq \braket{\partial_\lambda o(t)|\partial_\lambda o(t)})_c
    \end{equation}
}


\begin{proof}
    \begin{equation}
        \delta E^2\equiv \braket{o(t)|\HH(t)^2|o(t)}_c=\dot\lambda^2 G_{\lambda\lambda}+\O(\dot\lambda^4),
    \end{equation}
    where $\O(\dot\lambda^4)$ needs to be positive for any real-valued Hamiltonian. This comes from the fact, that it has instantaneous time-reversal symmetry.
\end{proof}

The conjecture only applies to unit fidelity protocols ($F(t)=1 \;\forall t\in[0,T_f]$) and can be extended to an arbitrary dimensional path.
