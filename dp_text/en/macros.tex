%%% This file contains definitions of various useful macros and environments %%%
%%% Please add more macros here instead of cluttering other files with them. %%%

%%% Minor tweaks of style

% These macros employ a little dirty trick to convince LaTeX to typeset
% chapter headings sanely, without lots of empty space above them.
% Feel free to ignore.
\makeatletter
\def\@makechapterhead#1{
  {\parindent \z@ \raggedright \normalfont
   \Huge\bfseries \thechapter. #1
   \par\nobreak
   \vskip 20\p@
}}
\def\@makeschapterhead#1{
  {\parindent \z@ \raggedright \normalfont
   \Huge\bfseries #1
   \par\nobreak
   \vskip 20\p@
}}
\makeatother

% This macro defines a chapter, which is not numbered, but is included
% in the table of contents.
\def\chapwithtoc#1{
\chapter*{#1}
\addcontentsline{toc}{chapter}{#1}
}

% Draw black "slugs" whenever a line overflows, so that we can spot it easily.
\overfullrule=1mm

%%% Macros for definitions, theorems, claims, examples, ... (requires amsthm package)

\theoremstyle{plain}
\newtheorem{thm}{Theorem}
\newtheorem{hypot}{Hypotheses}
\newtheorem{lemma}[thm]{Lemma}
\newtheorem{definition}{Definition}

\theoremstyle{plain}
\newtheorem{defn}{Definition}

\theoremstyle{remark}
\newtheorem*{cor}{Corollary}
\newtheorem{conjecture}{Conjecture}
\newtheorem*{rem}{Remark}
\newtheorem*{example}{Example}

%%% An environment for proofs

\newenvironment{myproof}{
  \par\medskip\noindent
  \textit{Proof}.
}{
\newline
\rightline{$\qedsymbol$}
}

%%% An environment for typesetting of program code and input/output
%%% of programs. (Requires the fancyvrb package -- fancy verbatim.)

\DefineVerbatimEnvironment{code}{Verbatim}{fontsize=\small, frame=single}


%%% Useful operators for statistics and probability
\DeclareMathOperator{\sign}{\textrm{sign}}
\renewcommand{\Im}{\textrm{Im}}
\newcommand{\Par}{\textrm{par}}

%%% Transposition of a vector/matrix
\newcommand{\T}[1]{#1^\top}

%%% Various math goodies
\newcommand{\maon}[1]{o(n^{#1})}
\newcommand{\abs}[1]{\left|{#1}\right|}
\newcommand{\isqr}[1]{\frac{1}{\sqrt{#1}}}

%%% Various table goodies
\newcommand{\pulrad}[1]{\raisebox{1.5ex}[0pt]{#1}}
\newcommand{\mc}[1]{\multicolumn{1}{c}{#1}}

\DeclareMathOperator{\Tr}{\textrm{Tr}}
\DeclareMathOperator\arctanh{arctanh}
\renewcommand{\d}{\ensuremath{\mathrm{d}}}
\newcommand{\D}{\ensuremath{\mathrm{D}}}
\newcommand{\pder}[2]{\frac{\partial #1}{\partial #2}}
\newcommand{\der}[2]{\frac{\mathrm{d} #1}{\mathrm{d} #2}}
\newcommand{\Der}[2]{\frac{\mathrm{D} #1}{\mathrm{d} #2}}

\newcommand{\M}{\mathcal{M}}
\renewcommand{\P}{\mathcal{P}}
\newcommand{\R}{\mathbb{R}}
\newcommand{\N}{\mathbb{N}}
\newcommand{\F}{\mathcal{F}}
\renewcommand{\T}{\mathbb{T}}
\newcommand{\TT}{\mathcal{T}}
\renewcommand{\O}{\mathcal{O}}

\newcolumntype{L}[1]{>{\raggedright\let\newline\\\arraybackslash\hspace{0pt}}m{#1}}
\newcolumntype{C}[1]{>{\centering\let\newline\\\arraybackslash\hspace{0pt}}m{#1}}
\newcolumntype{R}[1]{>{\raggedleft\let\newline\\\arraybackslash\hspace{0pt}}m{#1}}


\newcommand{\A}{\mathcal{A}}
\newcommand{\Id}{\mathbbm{1}}
\newcommand{\llambda}{{\bm\lambda}}
\renewcommand{\AA}{\mathcal{\widehat{A}}}
\newcommand{\U}{\hat{U}}
\renewcommand{\H}{\mathcal{H}}
\newcommand{\HH}{\hat{H}}
\newcommand{\J}{\hat{J}}
\newcommand{\kpsi}{\ket{\psi}}
\newcommand{\kphi}{\ket{\phi}}
\newcommand{\kpsit}{\ket{\psi(t)}}
\newcommand{\kpsilt}{\ket{\psi(\llambda(t))}}
\newcommand{\up}{\ket{\uparrow}}
\newcommand{\dn}{\ket{\downarrow}}
\newcommand{\ch}{\hat{\chi}}
\newcommand{\Schrodinger}{Schrödinger }
\newcommand{\PH}{\mathcal{PH}}
\newcommand{\Z}{\mathbb{Z}}
\newcommand{\Span}{\text{Span}}
\renewcommand{\Re}{\text{Re}}

\newcommand{\expsm}{e^{-\frac{i \omega}{2}\hat\sigma_y  t}}
\newcommand{\expsp}{e^{\frac{i \omega}{2}\hat\sigma_y  t}}
\newcommand{\UU}{\hat U}


\DeclareMathOperator{\spec}{\sigma}