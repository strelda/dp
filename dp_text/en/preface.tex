\chapter*{\red{Introduction}}
\addcontentsline{toc}{chapter}{Introduction}
One of the unsolved problems of the quantum physics are quantum computers. There are many mathematical problems, which are solvable in exponential time on computers with classical bits, but are solvable in polynomial time on quantum computers. Essentially you prepare some initial state of qubits (these might be quantum dots \citep{dots}, or more recently, Josephson junctions \citep{josephson} are used) and perform certain operations on them using \emph{quantum gates}. At the end you measure the qubits, causing the collapse of wave-function, and read the result. The first main problems in this area, is holding the superposition of qubits until all operations are performed. The second great problem is the quantum noise, either spontaneous emission of excited states, or interaction with the thermal basis of the surrounding. The impact of these effects can be seen on symmetrical experiment, in which we start with some state, let's say spin up. Perform any number of operation on it and then perform their inverse, leading to the same state, spin up. In ideal quantum computer, we would get the initial state with 100 \% accuracy. In reality, the state can collapse into different eigenstate, in this example it would be spin down. The \emph{percentage of getting the wanted result} is called the \emph{fidelity}.

This problem is of course more general. From mathematical point of view, in the example above we have interaction Hamiltonian between qubits, thermal basis and quantum gates. The interaction with gates can be described by some Hamiltonian element with free parameter. Changing this parameter influences the qubit and \emph{drives} it to some final state, which will be measured. The theory of quantum driving, as created by physicists in the second half of 20. century, uses mathematical formalism which sometimes lacks on precise definitions. It can be formalized in a language of differential geometry. The basics of differential geometry are presented in Chapter \ref{chap:mathIntro}. The theory of quantum driving itself is described in Chapter \ref{chap:driving}. 

The important question here is: “How to achieve the greatest \emph{final fidelity}, meaning \emph{how to prepare the state we want to prepare with the highest possible probability}?” During the driving one might add some energy to the qubit, which leads to its excitation and possibly destroying the superposition. This can be avoided by many methods. These methods are described in Chapter \ref{chap:typesOfDriving}. The surprising fact is that not every sequence of quantum gates leads to the same fidelity. For example if one starts with \emph{spin up}, applying the $X$ or $Y$ gate has the same effect. Both result in \emph{spin down}, because these gates just rotate the spin in a Bloch sphere around corresponding axis ($x$, resp. $y$).

For some special drivings, such as driving using small \emph{quenches} (quick, but small change in driving parameter), one might get interested in \emph{ground state manifold properties}.

To understand the general fidelity driving, a simple two level system is analyzed in Chapter \ref{chap:twoLevelSystem}. Some driving phenomena are demonstrated on the two analytically solvable protocols. Because with the Hamiltonian complexity, the driving complicates noticeably, it is important to understand the geometry of ground state manifolds first. The ground state manifold consists of all ground states of Hamiltonian with different driving parameter value. Special role plays geodesics. Some applications were developed in previous works, some are proposed here.



main goal: driving of systems