\chapter{Geometrization of quantum mechanics}
\label{appendixGEOM}
Differential geometry is believed to be the modern language of physics, and there is a strong urge to reformulate theories in this language. As introduction to the problem, see \cite{ashtekar_geometrical_1997}, \cite{ashtekar_geometry_1995}, or more mathematical work by \cite{molitor_exponential_2013}.

The whole thesis uses a classical formulation of quantum mechanics with some parts described using differential geometry. Here, the complete reformulation of quantum mechanics and the bridge between this theory and the already formulated are introduced. Reformulating the whole theory of quantum driving into the language of differential geometry might give some new insights, but it is beyond the scope of this thesis.


\section{From the projective Hilbert space to state manifolds}


Consider the Hilbert space $\H$ to be a space of \emph{bare states} and $\mathcal{S}$ to be the space of \emph{normalized bare states}. Physical observables are related to the \emph{space of rays}, defined as $\PH\coloneqq \mathcal{H}/U(1)$, for the factorization by elements of one-dimensional unitary group $U(1)$. This group consists of unitary transformations $e^{i\varphi}$ for $\varphi\in\R$, defining gauge symmetry between quantum states. $\PH$ is then considered to be the \emph{space of pure states}. We will consider the states to be normalized, leading to the \emph{space of normalized pure physical states}. 

It can be shown that $\PH$ is of K\"ahler structure, meaning it has two non-degenerate sesquilinear\footnote{Complex conjugated is the first input of the 2-form.} 2-forms embedded along with complex unit operator $J$, defining structure
$$(J, G, \Omega),$$
such that
\begin{equation}
    J^2=\Id.
\end{equation}
Any bracket of $\ket{\psi_1},\ket{\psi_2}\in \PH$ can be decomposed into real and imaginary part\citep{ashtekar_geometrical_1997}
\begin{equation}
    \braket{\psi_1|\psi_2}\equiv Q(\psi_1,\psi_2)=\frac{1}{2}G(\psi_1,\psi_2)-\frac{i}{2}\Omega(\psi_1,\psi_2).
\end{equation}

From braket, sesquilinearity goes that $G$ is symmetric and $\Omega$ antisymmetric form thus they can be uniquely written into one 2-form called \emph{Fubini-Study metric} with property
\begin{equation}
    G=\Re Q ;\qquad \Omega=\Im Q.
\end{equation}

Because $|\braket{\psi_1|\psi_2}|\in[0,1]$ we say, that the \emph{metric is measuring the geodesic distance on the Bloch sphere}. Here if we define
\begin{equation}
    |\braket{\psi_1|\psi_2}|=\cos^2 \frac{\theta}{2},
\end{equation}
we get $\d \theta= 2\d s=2\sqrt{|g_{\mu\nu}\d \llambda^\mu\d \llambda^\nu|}$, see \citet{cheng_quantum_2013}.

To write the metric in a standard form, we need to realize how our space looks like. For finite $(N+1)$-dimensional Hilbert space, one dimension is lost in the gauge transformation, leaving us with $N$-dimensional $\PH$. Another dimension is lost due to normalization, which is usually done by mapping to an n--dimensional complex sphere
$$CP^N= \left\{ \bm Z=(Z_0,Z_1,\dots,Z_N)\in \mathbb{C}^{N+1}/\{0\} \right\}\Big/ \{\bm Z\sim c\bm Z \text{ for } c\in \mathbb{C}\}.$$

Natural property of such complex spaces is splitting of its tangent space to holonomous and anholonomous part\footnote{The line over index means complex conjugation.}
$$\mathcal T^1_0\M=\Span\left\{\frac{\partial}{\partial Z_i}\right\}; \qquad \mathcal T^0_1\M=\Span\left\{\frac{\partial}{\partial Z_{\overline{i}}}\right\}.$$
For suitable distance $\d Z$, this can be used to define distance on state manifolds.


Distance on $\mathbb{C}^{n+1}$ is usually defined using Hermitian metric\footnote{Hermitian metric is by definition sesquilinear, as one would expect in quantum mechanics later on.} 
\begin{equation}
    \d s^2 = \d \overline{\Z} \otimes \d \Z.
\label{eq:metricdistancePH}
\end{equation}


% For \emph{normalized states} in quantum mechanics is $\d Z=(1, \ket{\d z})^T$, which plugged into Eq. \ref{eq:metricdistancePH} yields
% \begin{equation}
%     \d s^2 = 1-\left|\braket{\psi+\delta \psi|\psi}\right|^2.
%     \label{eq:distanceInPH}
% \end{equation}


\section{Restriction to eigenstate manifolds}
In quantum mechanics, one can examine a Hamiltonian $\HH(\llambda)$, for some parameter $\llambda\in \mathcal U\subset \R^n$. At every point $\llambda$ we get projective Hilbert space $\PH(\llambda)$. This creates a fiber structure space, in which there are some sections with interesting physical applications. Some of these sections are \emph{eigenstate manifolds}, defined by setting only one non-zero coefficient $Z_k$ in eigenbasis $\ket{\psi}=\sum_{k=0}^n Z_k \ket{k}$. From normalization goes automatically $Z_k=1$. The distance on these manifolds is, as derived in Eq. \ref{eq:distanceOnM0},
\begin{equation}
    \begin{split}
        \d s^2 &= \textcolor{gray}{1}-\bra{k+\delta k}\textcolor{gray}{k\rangle \langle k}\ket{k+\delta k}=\textcolor{gray}{1}-\bra{k+\delta k}\textcolor{gray}{\big(\Id-\sum_{j\neq k} \ket{j}\bra{j}\big)}\ket{k+\delta k} \\
        &= \textcolor{gray}{\sum_{j\neq k}}\bra{k+\delta k}\textcolor{gray}{j\rangle \langle j}\ket{k+\delta k}.
        \label{eq:dsDerivation}
    \end{split}
\end{equation}
Using the \Schrodinger equation $\textcolor{blue}{\HH}\ket{k}=\textcolor{teal}{E_k} \ket{k}$, distributivity of derivative and projection to some state $\textcolor{gray}{\bra{j}}$, we get
\begin{equation}
    \begin{split}
        \textcolor{blue}{\HH}\ket{k} &= \textcolor{teal}{E_k}\ket{k}\\
        \textcolor{blue}{(\delta \HH)}\ket{k} +\textcolor{blue}{\HH}\ket{k+\delta k} &=\textcolor{teal}{(\delta E_k)}\ket{k} +\textcolor{teal}{E_k}\ket{k+\delta k}\\
         \textcolor{gray}{\bra{j}}\big(\textcolor{blue}{\delta \HH}-\textcolor{teal}{\delta E_k}\big)\ket{k}&=\textcolor{gray}{\bra{j}}\big(\textcolor{teal}{E_k}-\textcolor{blue}{\HH}\big)\ket{k+\delta k}=\textcolor{gray}{\bra{j}}\big(\textcolor{teal}{E_k}-\textcolor{blue}{E_j}\big)\ket{k+\delta k}.
    \end{split}
\end{equation}
We can set
%\footnote{\textcolor{red}{Can it be done only for $E_0$? It does not make sence generally, because $E=E(\llambda)$, even $E_0=E_0(\llambda)$}} 
$\textcolor{blue}{\delta E_k}=0$, leading for $j\neq k$ to
\begin{equation}
    \frac{\textcolor{gray}{\bra{j}}\textcolor{blue}{\delta \HH}\ket{k}}{(\textcolor{teal}{E_k}-\textcolor{blue}{E_j})^2}=\textcolor{gray}{\langle j}\ket{k+\delta k}.
    \label{eq:braket_k,deltaj}
\end{equation}
Plugging to Equation \ref{eq:dsDerivation} and considering $\HH=\HH(\llambda)$, we get metric on a ground state manifold
\begin{equation}
    \d s^2 = \Re\textcolor{gray}{\sum_{j\neq k}} \frac{\bra{0}\textcolor{blue}{\partial_\mu \HH \textcolor{gray}{\ket{j}}}\textcolor{gray}{\bra{j}}\textcolor{blue}{\partial_\nu \HH}\ket{0}}{(\textcolor{teal}{E_k}-\textcolor{blue}{E_j})^2}  \d \llambda^\mu\d \llambda^\nu
    \label{eq:distanceonM}
\end{equation}


Definition of the k--state manifold is then
\begin{equation}
    g_{\mu\nu}^{(k)} = \Re \sum_{j\neq k}\frac{\braket{k|\pder{\HH(\llambda)}{\lambda^\mu}|j}\braket{j|\pder{\HH(\llambda)}{\lambda^\nu}|k}}{(E_k-E_j)^2}.
    \label{eq:metrictensork}
\end{equation}
The Fubini-Study metric on the eigenstate manifold is sometimes called \emph{Geometric tensor}. 
