\chapter*{Conclusion}
\addcontentsline{toc}{chapter}{Conclusion}
This thesis presents the theory of quantum state driving in finite Hamiltonian systems. The well known claims were reformulated into more rigorous theorems and definitions. This mathematical approach has potential to reduce the barrier for any mathematically based scientist when trying to approach this theory. The playground in the form of fiber space was constructed, and the geometry of energy states reformulated on it. The idea of "advantageous driving" with high fidelity was scrutinized, especially as adiabatic, close-adiabatic and counter-diabatic driving.

The correspondence to a damped harmonic oscillator in the fidelity behavior was shown on a simple two-level system. In the case of geodesic driving, the fidelity oscillates with constant frequency and periodically becomes one. For linear driving, the fidelity has essentially two regimes — fast transport regime, described by the semiclassical Landau-Zener formula, and close adiabatic regime, described by APT. In both models, the fidelity is excited when the difference between energy levels gets small. These excitations lead to damped oscillations.

In the LMG model, the ground state manifold was analyzed. The Riemannian manifold characteristics were calculated along with their implications. These are the ground state manifold geodesics and the coordinates of diabolic points in the parametric space, dependent on the Hamiltonian dimension. In three dimensions, this was done analytically, proving the existence of diabolic points. For higher dimensions, numerical methods were used. The transport using quenches was numerically demonstrated, showing the transformation to adiabatic transport by shortening the quenches to zero.

Many questions still lay unsolved, and some were newly opened. For example: "What are the possibilities for quantum quench transport?". "What is the correspondence of energy variance and driving fidelity?". Further on, from the LMG model, the proposed analytical formula for diabolic points coordinates remains to be proven analytically, along with the number of these points.

This thesis provides a significant amount of numerical analysis on two quantum models, which will hopefully serve for future discoveries in this area of physics. The state manifold analyses might lead the search for better fidelity protocols, or maybe the geodesics will be found to be somewhat "most stable protocols" for counter-diabatic driving. Either way, the possibilities are yet open and not known.