\section{Transport using quenches}
In Chapter \ref{chap:groundStateManifoldDriving}, the transport using quenches was introduced. This transport also resembles some ground-state manifold characteristics. Performing quench from $(\lambda_i;\chi_i)=(0;0)$ to $(\lambda;\chi)$, the fidelity is small if the final coordinates are below separatrix. When it is behind the \greenn{minimal line}, the fidelity decreases dramatically, as can be seen on Fig. \ref{fig:quenchFidelityFrom00}.

Transport using quenches can be observed in Fig. \ref{fig:plotsFidelityQuenches}. The quenches are performed around paths with $\int_{a}^{b} \d s=\text{const}$ between every two neighboring points on ground-state manifold geodesic. If the system is measured periodically, the quenches jump smaller distances when the coordinates in parametric space are closer to the degeneracy. Decreasing time step $\Delta t$ has no effect on the relative fidelity of quenches during the evolution but affects their absolute fidelity. As one would expect that for $\Delta t\rightarrow 0$, transport becomes adiabatic, and the fidelity at any time will become $1$. See that the shape of the curves looks similar in plots in the columns, but their magnitude decreases.

\begin{figure}[h]
    \centering
    \includegraphics[scale=1.2]{../img/quenchFidelityFrom00.pdf}
    \caption{Fidelity of quench from $(\lambda_i;\chi_i)=(0;0)$ to the coordinate $(\lambda;\chi)$. \greenn{Minimal line (green)} is plotted for comparison.}
    \label{fig:quenchFidelityFrom00}    
\end{figure}

% \begin{figure}[h]
%     \centering
%     \includegraphics[scale=1.2]{../img/equidistantPointsOnPath.pdf}
%     \caption{Equidistant points on geodesics of the ground state manifold.}
%     \label{fig:equidistantPointsOnPath}    
% \end{figure}

\begin{figure}[h]
    \centering
    \includegraphics[scale=1.2]{../img/bg123.pdf}
    \includegraphics[scale=1.2]{../img/plotsFidelityQuenches.pdf}
    \caption{Fidelity for sequential quenches along geodesics (see green lines on top). Left (right) column corresponds to lower (upper) geodesic. Time steps from top are $\Delta t\in \{0.03,0.1,0.5\}$. Time difference between points in the plot on top is $\Delta t=0.5$.}
    \label{fig:plotsFidelityQuenches}    
\end{figure}


