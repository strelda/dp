\chapter{APT}
From \cite{Rigolin2008}.
One important thing to bare in mind is \emph{locality of variables}. Let's call variable $V(s)$ \emph{\bluee{local}} if it depends only on infinitesimal sirrounding of $s$. These variables will be in shades of blue and non-local variables in shades of red.

The power series will be derived using a small parameter $v=1/T$. Starting again with
\begin{equation}
    \red{\ket{\Psi(s)}}=\sum_{p=0}^\infty v^p \red{\ket{\Psi^{(p)}(s)}},
    \label{eq:mainSeries}
\end{equation}

for 
\begin{equation}
    \red{\ket{\Psi^{(p)}(s)}}=\sum_{n=0} e^{-\frac{i}{v} \reddd{\omega_n(s)}}e^{i\reddd{\gamma_n(s)}}\redd{b_n^{(p)}(s)}\bluee{\ket{n(s)}}.
\end{equation}
Here the
\begin{align}
    \reddd{\omega_n(s)} &\coloneqq \frac{1}{\hbar}\int_0^s \bluee{E_n(s')}\d s'\\
    \reddd{\gamma_n(s)} &\coloneqq i\int_0^s \bluee{\braket{n(s')|\frac{\d}{\d s'}n(s')}}\d s' \equiv i\int_0^s \bluee{M_{nn}(s')}\d s'
    \label{eq:gammadef}
\end{align}
are so-called \redd{dynamical} resp. \redd{Berry (geometric) phase} and $\bluee{\ket{n(s)}}$ are solution to
\begin{equation}
    \bluee{\HH(s)\ket{n(s)}}=\bluee{E_n(s) \ket{n(s)}}.
\end{equation}
Variables $\reddd{\omega_n(s)}$ and $\reddd{\gamma_n(s)}$ are defined using integration over the whole protocol, therefore there are \emph{\red{non/local variables}}.
The problem now lies in determining $\redd{b_n^{(p)}(s)}$, which is also \red{nonlocal}. Because it depends on its relative \reddd{geometric} and \reddd{dynamical phase} to other \bluee{energy levels}, lets write it as a series
\begin{equation}
    \redd{b_n^{(p)}(s)}=\sum_{m=0} e^{\frac{i}{v}\reddd{\omega_{nm}(s)}}e^{-i\reddd{\gamma_{nm}(s)}}\blue{b_{nm}^{(p)}(s)},
\end{equation}
where $\reddd{\omega_{nm}} \coloneqq \reddd{\omega_m}-\reddd{\omega_n}$, $\reddd{\gamma_{nm}} \coloneqq \reddd{\gamma_m}-\reddd{\gamma_n}$.  The reason for \blue{locality} of $\blue{b_{nm}^{(p)}(s)}$ will be clear soon.

Inserting all to original series \ref{eq:mainSeries}, we get
\begin{equation}
    \red{\ket{\Psi(s)}}=\sum_{n,m=0}\sum_{p=0}^\infty v^p e^{-\frac{i}{v}\reddd{\omega_m(s)}}e^{i\reddd{\gamma_m(s)}}\redd{b_{nm}^{(p)}(s)}\bluee{\ket{n(s)}}.
    \label{eq:solve0}
\end{equation}

Because the initial state is eigenstate, we get initial conditions $\blue{b_{nm}^{(0)}(s)=0}$. In addition, one can rewrite equation \ref{eq:solve0} to the iteratively solvable form
\begin{equation}
    \frac{i}{\hbar}\bluee{\Delta_{nm}(s)}\blue{b_{nm}^{(p+1)}(s)}+\blue{\dot b_{nm}^{(p)}(s)}+\bluee{W_{nm}(s)} \blue{b_{nm}^{(p)}(s)}+\sum_{k=0,k\neq n}\bluee{M_{nk}(s)}\bluee{b_{km}^{(p)}(s)}=0,
    \label{eq:bSolution}
\end{equation}
for $\bluee{\Delta_nm(s)}\coloneqq \bluee{E_m-E_n}$, $\bluee{W_{nm}(s)}\coloneqq \bluee{M_{nn}(s)}-\bluee{M_{mm}(s)}$, where $\bluee{M_{mn}}$ is defined in Eq. \ref{eq:gammadef}. We can see that $\blue{b_{mn}^{(p)}}$, as a solution to Eq. \ref{eq:bSolution},\textbf{ only depends on difference between energy levels, eigenstates during the path and their directional derivatives. Not on the path itself}. All of those are easily obtained, once the driving path is prescribed.

    
