\chapter{Lipkin-Meshkov Glick model}
The Lipkin-Meshkov Glick model is a simple first order transition Hamiltonian and can be written in a form
\begin{equation}
    \HH=\J_3-\frac{\tilde\lambda(t)}{2j}\left(\J_1+\tilde\chi(\J_3+j\Id)\right)^2
    \label{eq:firstOrderTransitionHamiltonianOriginal}
\end{equation}
for $\J=(\J_1,\J_2,\J_3)$ angular momentum operator, $\tilde\chi\in\R$ and $\tilde\lambda(t)$ some real time dependent parameter. The aim is to diagonalize this Hamiltonian to find the spectrum. To make driving parameters less entwined, we will instead use the Hamiltonian
\begin{equation}
    \HH=\J_3+\lambda\hat V_1 +\chi \hat V_2+\chi^2 \hat V_3,
    \label{eq:firstOrderTransitionHamiltonian}
\end{equation}
for
\begin{align}
    \hat V_1 &=-\frac{1}{2j}\J_1^2\\
    \hat V_2 &= -\frac{1}{2j}\left[\J_1(\J_3+j\Id)+(\J_3+j\Id)\J_1\right]\\
    \hat V_3 &= -\frac{1}{2j}(\J_3+j\Id)^2
\end{align}



Using the eigenbasis $\{\ket{j,m}\}$ for $j$ angular momentum quantum number and $m$ its projection to the direction $\J_3$ and defining
\begin{equation}
    \J_\pm\coloneqq\frac{1}{2}(\J_1\pm i\J_2),
\end{equation}
we get matrix elements
\begin{align}
    \braket{j'm'|\J^2|jm} &= j(j+1)\delta_{j'j}\delta_{m'm}\\
    \braket{j'm'|\J_3|jm} &= m \delta_{j'j}\delta_{m'm}\\
    \braket{j'm'|\J_\pm|jm} &= \sqrt{(j\mp m)(j\pm m+1)}\delta_{j'j}\delta_{m'm\pm 1},
\end{align}
where $\delta_{a'b}$ is Kronecker delta. Using
\begin{align}
        \left(\J_1+\chi(\J_3+j)\right)^2 &= \textcolor{purple}{\J_1^2} +\chi^2 (\J_3^2+j^2\Id+2j\J_3)+\chi(\textcolor{blue}{\J_1}\J_3+\J_3\textcolor{blue}{\J_1})+2\chi j\textcolor{blue}{\J_1}\\
        \textcolor{purple}{\J_1^2}&= \frac{1}{4}(\J_++\J_-)^2= \frac{1}{4}(\J_+^2+\J_-^2+\textcolor{violet}{\J_+\J_-}+\textcolor{violet}{\J_-\J_+})\\ 
        \textcolor{violet}{\J_\pm\J_\mp}&=\J^2-\J_3^2 \mp \J_3\\
        \textcolor{blue}{\J_1}&=(\J_+ +\J_-),
\end{align}
we get pentadiagonal matrix representation of $\HH$. 

From now on, $j=N/2$ will be used, where $N$ is number of particles in the system. Because every particle has only two eigenstates -- spin up/down (we will call them \emph{qubits}), all of them are therefore aligned in one direction.
% \begin{equation}
%     \HH=\begin{pmatrix}
%         h_a   &h_b   &h_c   &0     &\dots &0\\
%         h_b   &\ddots&\ddots&\ddots&\ddots&\vdots\\
%         h_c   &\ddots&\ddots&\ddots&\ddots&0\\
%         0     &\ddots&\ddots&\ddots&\ddots&h_c\\
%         \vdots&\ddots&\ddots&\ddots&\ddots&h_b\\
%         0     &\dots &0     &h_c   &h_b   &h_a\\
%     \end{pmatrix}
% \end{equation}


\section{Hamiltonian analysis}
The behavior of cases $N=1$ and $N=2$ is not characteristic for our Hamiltonian and has no physical meaning in this case, therefore the properties of $N=3$ case will be discussed at first and afterwards generalized to arbitrary $N$ along with the limit $N\rightarrow\infty$. Due to complexity of chosen Hamiltonian, only some aspects are possible to calculate analytically, but mostly the numerical analysis is needed.



\subsection{Case N=3}
The lowest dimension with characteristics common to higher $N$ is 3 with Hamiltonian in eigenbasis
\begin{equation}
    \HH=\left(
        \begin{array}{cccc}
         -\frac{ \lambda +6}{4} & -\frac{\chi }{2 \sqrt{3}} & -\frac{\lambda }{2 \sqrt{3}} & 0 \\
         -\frac{\chi }{2 \sqrt{3}} & \frac{ \left(-7 \lambda -4 \chi ^2-6\right)}{12} & -\chi  & -\frac{\lambda }{2 \sqrt{3}} \\
         -\frac{\lambda }{2 \sqrt{3}} & -\chi  & \frac{ \left(-7 \lambda -16 \chi ^2+6\right)}{12} & -\frac{5 \chi }{2 \sqrt{3}} \\
         0 & -\frac{\lambda }{2 \sqrt{3}} & -\frac{5 \chi }{2 \sqrt{3}} & -\frac{\lambda }{4}-3 \chi ^2+\frac{3}{2} \\
        \end{array}
        \right).
\end{equation}
Spectrum of this Hamiltonian can be written in analytically using some substitutions $A,B,C,D$, see Appendix \ref{appendix1}, as
\begin{align}
        E_0 &= \frac{1}{12} \left(G-F-\frac{\sqrt{D-E}}{2}\right)
        \label{eq:N=3_en0}\\
        E_1 &= \frac{1}{12}  \left(G-F+\frac{\sqrt{D-E}}{2}\right)
        \label{eq:N=3_en1}\\
        E_2 &= \frac{1}{12} \left(G+F-\frac{\sqrt{D+E}}{2}\right)
        \label{eq:N=3_en2}\\
        E_3 &= \frac{1}{12}  \left(G+F+\frac{\sqrt{D+E}}{2}\right).
        \label{eq:N=3_en3}
\end{align}
Eigenvectors can also be written analytically, but due to their compexity its better to analize them only numerically. Sections $\lambda=1$ and $\chi=1$ are shown in figures \ref{fig:N=3_energiesl},\ref{fig:N=3_energiesc}
\begin{figure}[h]
    \centering
    \includegraphics{../img/N=3_energiesl.pdf}
    \caption{Energy for the case $N=3$ Hamiltonian, section $\lambda=1$}
    \label{fig:N=3_energiesl}
\end{figure}
\begin{figure}[h]
    \centering
    \includegraphics{../img/N=3_energiesc.pdf}
    \caption{Energy for the case $N=3$ Hamiltonian, section $\chi=1$}
    \label{fig:N=3_energiesc}
\end{figure}

From equations \ref{eq:N=3_en0}, \ref{eq:N=3_en1} can be seen that $E_0=E_1$ for $D=E$, which for real values $\lambda,\;\chi$ has two solutions
$$(\lambda_d,\pm \chi_d)=\left(-\frac{1}{2},\pm\sqrt{\frac{3}{5}}\right).$$
Point-like characteristics corresponds to Theorem \ref{thm:n-2}, which states that Hamiltonian driven by two real parameters can be degenerated only on 0-dimensional manifolds. 

If the energy spectrum is degenerate and metric tensor diverges, see individual elements on fig. \ref{fig:N=3_g}, its determinant also diverges, as shown on fig. \ref{fig:N=3_gDivenrgence}, along with Christoffel symbols elements, see fig. \ref{fig:N=3_G}. Note that metric tensor determinant is positive definite, thus the manifold is Riemannian. Further on it reflects symmetry $\chi\leftrightarrow-\chi$, except for off-diagonal elements, i.e. $g_{12}$, $\Gamma_{121}$, $\Gamma_{211}$, $\Gamma_{222}$, which switches their sign.
\begin{figure}
    \centering
    \begin{tabular}{cc}
    \subcaptionbox{$\arctan(g_{11})$}{\includegraphics{../img/N=3_g11.pdf}} \\
    \subcaptionbox{$\arctan(g_{12})$}{\includegraphics{../img/N=3_g12.pdf}}\\
    \subcaptionbox{$\arctan(g_{22})$}{\includegraphics{../img/N=3_g22.pdf}}\\
    \end{tabular}
\includegraphics{../img/N=3_barA.pdf}
    \caption{Arcustangens of metric tensor elements for the case $N=3$}
    \label{fig:N=3_g}
\end{figure}

\begin{figure}[h]
    \centering
    \includegraphics{../img/N=3_gDivergence.pdf}
    \caption{Metric determinant in a parameter space for $N=3$.}
    \label{fig:N=3_gDivenrgence}    
\end{figure}


\begin{figure}[h]
    \centering
    \begin{tabular}{cc}
    \subcaptionbox{$\arctan(\Gamma_{111})$}{\includegraphics{../img/N=3_G111.pdf}} &
    \subcaptionbox{$\arctan(\Gamma_{211})$}{\includegraphics{../img/N=3_G211.pdf}} \\
    \subcaptionbox{$\arctan(\Gamma_{121})$}{\includegraphics{../img/N=3_G121.pdf}} &
    \subcaptionbox{$\arctan(\Gamma_{221})$}{\includegraphics{../img/N=3_G221.pdf}}\\
    \subcaptionbox{$\arctan(\Gamma_{122})$}{\includegraphics{../img/N=3_G122.pdf}} &
    \subcaptionbox{$\arctan(\Gamma_{222})$}{\includegraphics{../img/N=3_G222.pdf}}\\
    \end{tabular}
\includegraphics{../img/N=3_barA.pdf}
    \caption{Arcustangens of Christoffel symbols for the case $N=3$}
    \label{fig:N=3_G}
\end{figure}

Due to metric tensor degeneracy, the space is not geodesically maximal. To see that singularity is not only coordinate one, by calculating the Ricci curvature we see (fig. \ref{fig:N=3_Ricci}), that it diverges in points $(\lambda_d;\pm\chi_d)$ (singularities of the metric tensor), see sections in $\chi$-direction on fig. \ref{fig:N=3_Ricci_section}, proving the singularity is not only coordinate, but \emph{physical}. Coordinates of those minimas can be seen on fig. \ref{fig:N=3_RicMinimas}.

This means our ground state manifold is geodesically incomplete and according to \ref{thm:hopf-Rinow_modified} there exist some "event horizon" around singularities $(\lambda_d,\pm \chi_d)$. 
\begin{figure}[h]
    \centering
    \includegraphics{../img/N=3_Ricci.pdf}
    \caption{Arcustanges of Ricci curvature for the case $N=3$.}
    \label{fig:N=3_Ricci}
\end{figure}
\begin{figure}[h]
    \centering
    \includegraphics{../img/N=3_Ricci_section.pdf}
    \caption{Ricci curvature, section in $\lambda\in\{0.5;1;1.5\}$.}
    \label{fig:N=3_Ricci_section}
\end{figure}
\begin{figure}[h]
    \centering
    \includegraphics{../img/N=3_RicMinimas.pdf}
    \caption{Minimas in Ricci curvature. Case $N=3$.}
    \label{fig:N=3_RicMinimas}    
\end{figure}

This can be seen more clearly from geodesics, i.e. by solving initial value problem with conditions
$$(\lambda(t_i);\chi(t_i))=(\lambda_i;\chi_i)$$
$$\left(\der{\lambda(t)}{t};\der{\chi(t)}{t}\right)\Bigg|_{t_i}=(\lambda'_i;\chi'_i).$$

Results for these geodesics starting at $(\lambda;\chi)=(0;0)$, $(\lambda',\chi')=(\cos\theta;\sin\theta)$ for $\theta\in [-0.63;0.63]$ and $\theta\in [\pi-0.225;\pi+0.225]$ with step $0.01$, see fig \ref{fig:N=3_geodesics}. Other values $\theta$ result in close approach of the geodesics to singularity making calculations numerically unstable. The fact that geodesics lean towards singularities is well known from the theory of General Relativity (GR). The main difference here is, that our "test particle" seems to be partially repulsed from the singularity, see charge attraction vs. repulsion intuition in fig. \ref{fig:geodesicsinGR}. \textcolor{blue}{This is caused by the fact, that the Hamiltonian consists of more elements from which some are "repulsive" and some "attractive". } Also from the singularity to the right, there is an area with huge negative Ricci curvature, which the geodesics try to avoid resulting in them going around the singularity, which is shorter than crossing it.



Another interesting initial value problem is $(\lambda_i;\chi_i)=(-1;\chi_i)$, $(\lambda',\chi')=(1;0)$, which is for $\chi_i\in(0;1)$ with step $0.05$ shown in fig. \ref{fig:N=3_geodesics_lambdaIn=-1}, where numerically unstable geodesics were skipped.

\begin{figure}[h]
    \centering
    \includegraphics{../img/N=3_geodesics.pdf}
    \caption{Geodesics starting from $(\lambda_i;\chi_i)=(0;0)$ with $(\lambda'_i;\chi'_i)=(\cos\theta;\sin\theta)$. Case $N=3$.}
    \label{fig:N=3_geodesics}    
\end{figure}
\begin{figure}[h]
    \centering
    \includegraphics[width=0.6\textwidth]{../img/geodesicsinGR.png}
    \caption{Comparing geodesics with repulsing (blue) and attracting (red) metric tensor divergence in the spherical symmetrical space. }
    \label{fig:geodesicsinGR}
\end{figure}

\begin{figure}[h]
    \centering
    \includegraphics{../img/N=3_geodesics_lambdaIn=-1.pdf}
    \caption{Geodesics starting from $(\lambda_i;\chi_i)=(-1;\chi_i)$, $(\lambda',\chi')=(1;0)$, for $\chi_i\in(0;1)$ with step $0.05$. Numerically unstable geodesics were skipped. Case $N=3$.}
    \label{fig:N=3_geodesics_lambdaIn=-1}    
\end{figure}



\newpage
\section{Limit \texorpdfstring{$N\rightarrow \infty$}{N->infty}}
The limit $N\rightarrow \infty$ can be applied on the Hamiltonian in eq. \ref{eq:firstOrderTransitionHamiltonian} after Holstein-Primakoff mapping for bosonic operators
\begin{equation}
    \mathcal{H}\coloneqq\lim_{j\rightarrow\infty}\frac{\HH}{2j},
\end{equation}
resulting in classical Hamiltonian
\begin{equation}
    \begin{split}
        \mathcal{H}(x,p)=&-\frac{1}{2}+\frac{1-\lambda}{2}x^2+\frac{\lambda-\chi^2}{4}x^4-\frac{\chi x^3}{2}\sqrt{2-x^2-p^2}-\frac{\chi^2}{4}p^4\\
        &+\frac{p^2}{4}\left[2+(\lambda-2\chi^2)x^2-2\chi x\sqrt{2-x^2-p^2}\right].
    \end{split}
    \label{eq:HamiltonianClassicalLimit}
\end{equation}




Finding minimas in its derivatives, see \textcolor{blue}{reference to Felipe publication, it will be there hopefully and it has no point to solve it here} we get \emph{separatrix}
\begin{equation}
    \chi^2=\frac{\lambda-1}{\lambda-2},
    \label{eq:separatrix}
\end{equation}
which represents phase transition in the limit $N\rightarrow \infty$. In our case, the transition is of first order everywhere, except in $(\lambda;\chi)=(1;0)$, where it has order two. The separatrix is shown on fig. \ref{fig:transitionCompare} compared to minimum between the ground state and first excited state $E_1-E_0$ for $N=3$ case. With increasing $N$, it slowly converges to the separatrix line.

\begin{figure}[h]
    \centering
    \includegraphics{../img/infiniteN_transitionCompare.pdf}
    \caption{First order transition -- Separatrix (red), second order transition (blue point) compared to minimum between the ground state and first excited state in N=3 case (black, dashed).}
    \label{fig:transitionCompare}    
\end{figure}

\section{Arbitrary $N$}
The case $N=3$ was deliberately chosen to show the lowest dimension with characteristics, which generalize to higher dimensions, see energy spectrum sections on fig. \ref{fig:N=10_energiesl}, \ref{fig:N=10_energies2} for the $N=10$ case. \textcolor{red}{Between every two energy levels is at least one avoided crossing.} It was shown analytically, that for the case $N=3$ there is a Metric tensor singularity. In fact, this singularity is present for every $N$ and for the first 14 cases was shown numerically, that it moves along separatrix, see fig. \ref{fig:singularitiesRegr}. Positions of those points are not exact due to complicated numerics around singularity.
 


\begin{figure}[h]
    \centering
    \includegraphics{../img/N=10_energiesl.pdf}
    \caption{Energy spectrum as function of $\chi$, for $\lambda=1$ and $N=10$.}
    \label{fig:N=10_energiesl}    
\end{figure}
\begin{figure}[h]
    \centering
    \includegraphics{../img/N=10_energies2.pdf}
    \caption{Energy spectrum as function of $\lambda$, for $\chi=1$ and $N=10$.}
    \label{fig:N=10_energies2}    
\end{figure}

\begin{figure}[h]
    \centering
    \includegraphics{../img/divPosition.pdf}
    \caption{Singularities in the metric tensor for of $N\in\{3,4,\dots ,14\}$ and separatrix according to eq. \ref{eq:separatrix}.}
    \label{fig:singularitiesRegr}    
\end{figure}


