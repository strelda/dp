\chapter{Lipkin-Meshkov Glick model}
The Lipkin-Meshkov Glick model is a simple first order transition Hamiltonian and can be written in a form
\begin{equation}
    \HH=\J_3-\frac{\tilde\lambda(t)}{2j}\left(\J_1+\tilde\chi(\J_3+j\Id)\right)^2
    \label{eq:firstOrderTransitionHamiltonianOriginal}
\end{equation}
for $\J=(\J_1,\J_2,\J_3)$ angular momentum operator, $\tilde\chi\in\R$ and $\tilde\lambda(t)$ some real time dependent parameter. The aim is to diagonalize this Hamiltonian to find the spectrum. To make driving parameters less entwined, we will instead use the Hamiltonian
\begin{equation}
    \HH=\J_3+\lambda\hat V_1 +\chi \hat V_2+\chi^2 \hat V_3,
    \label{eq:firstOrderTransitionHamiltonian}
\end{equation}
for
\begin{align}
    \hat V_1 &=-\frac{1}{2j}\J_1^2\\
    \hat V_2 &= -\frac{1}{2j}\left[\J_1(\J_3+j\Id)+(\J_3+j\Id)\J_1\right]\\
    \hat V_3 &= -\frac{1}{2j}(\J_3+j\Id)^2
\end{align}



Using the Spherical harmonics basis $\{\ket{j,m}\}$ for quantum numbers $j$ as angular momentum and $m$ its projection to the direction of $\J_3$ and defining
\begin{equation}
    \J_\pm\coloneqq\frac{1}{2}(\J_1\pm i\J_2),
\end{equation}
we get matrix elements
\begin{align}
    \braket{j'm'|\J^2|jm} &= j(j+1)\delta_{j'j}\delta_{m'm}\\
    \braket{j'm'|\J_3|jm} &= m \delta_{j'j}\delta_{m'm}\\
    \braket{j'm'|\J_\pm|jm} &= \sqrt{(j\mp m)(j\pm m+1)}\delta_{j'j}\delta_{m'm\pm 1},
\end{align}
where $\delta_{a'b}$ is Kronecker delta. Hamiltonian in eq. \ref{eq:firstOrderTransitionHamiltonian} can then be written as
\begin{equation}
\begin{split}
        % \left(\J_1+\chi(\J_3+j)\right)^2 &= \textcolor{purple}{\J_1^2} +\chi^2 (\J_3^2+j^2\Id+2j\J_3)+\chi(\textcolor{blue}{\J_1}\J_3+\J_3\textcolor{blue}{\J_1})+2\chi j\textcolor{blue}{\J_1}\\
        % \textcolor{purple}{\J_1^2}&= \frac{1}{4}(\J_++\J_-)^2= \frac{1}{4}(\J_+^2+\J_-^2+\textcolor{violet}{\J_+\J_-}+\textcolor{violet}{\J_-\J_+})\\ 
        % \textcolor{violet}{\J_\pm\J_\mp}&=\J^2-\J_3^2 \mp \J_3\\
        % \textcolor{blue}{\J_1}&=(\J_+ +\J_-),\\
        \HH &= J_3-\frac{\lambda}{8j}(J_++J_-)^2-\frac{\chi}{4j}\left[(J_++J_-)(J_3+j\Id)+(J_3+j\Id)(J_++J_-)\right]\\
        &-\frac{\chi^2}{2j}(J_3+j\Id)^2,
\end{split}
\end{equation}
which has pentadiagonal matrix representation. During whole \textcolor{red}{thesis} $j=N/2$ will be used. 

\section{Hamiltonian analysis}
The behavior of dimensions $N=1$ and $N=2$ are fundamentally different from higher dimensions, thus the discussion will start at $N=3$. Then the limit $N\rightarrow\infty$ will be taken along with generalization of some characteristics to arbitrary dimension. Due to complexity of our Hamiltonian, it is not possible to prove every statement analytically and some numerical methods, mostly in Mathematica and Python, are needed.



\subsection{Case N=3}
The lowest dimension behaving similar to higher $N$ is 3 with Hamiltonian
\begin{equation}
    \HH=\left(
        \begin{array}{cccc}
         -\frac{ \lambda +6}{4} & -\frac{\chi }{2 \sqrt{3}} & -\frac{\lambda }{2 \sqrt{3}} & 0 \\
         -\frac{\chi }{2 \sqrt{3}} & \frac{ \left(-7 \lambda -4 \chi ^2-6\right)}{12} & -\chi  & -\frac{\lambda }{2 \sqrt{3}} \\
         -\frac{\lambda }{2 \sqrt{3}} & -\chi  & \frac{ \left(-7 \lambda -16 \chi ^2+6\right)}{12} & -\frac{5 \chi }{2 \sqrt{3}} \\
         0 & -\frac{\lambda }{2 \sqrt{3}} & -\frac{5 \chi }{2 \sqrt{3}} & -\frac{\lambda }{4}-3 \chi ^2+\frac{3}{2} \\
        \end{array}
        \right).
\end{equation}
Spectrum of this Hamiltonian can be calculated analytically using some substitutions $A,B,C,D$, see Appendix \ref{appendix1}, as
\begin{align}
        E_0 &= \frac{1}{12} \left(G-F-\frac{\sqrt{D-E}}{2}\right)
        \label{eq:N=3_en0}\\
        E_1 &= \frac{1}{12}  \left(G-F+\frac{\sqrt{D-E}}{2}\right)
        \label{eq:N=3_en1}\\
        E_2 &= \frac{1}{12} \left(G+F-\frac{\sqrt{D+E}}{2}\right)
        \label{eq:N=3_en2}\\
        E_3 &= \frac{1}{12}  \left(G+F+\frac{\sqrt{D+E}}{2}\right).
        \label{eq:N=3_en3}
\end{align}
Eigenvectors can also be written analytically, but writing them here would be redundant. On sections $\lambda=1$ and $\chi=1$, see figures \ref{fig:N=3_energiesl} resp. \ref{fig:N=3_energiesc}, can be seen general behaviour of the spectrum, where energies get close to each other somewhere around the center of our coordinate system $(\lambda;\chi)$ and then separate monotonously to never meet again. In addition, the spectrum is symmetrical for $\chi\leftrightarrow -\chi$.
\begin{figure}[H]
    \centering
    \includegraphics{../img/N=3_energiesl.pdf}
    \caption{Energy for the case $N=3$, section $\lambda=1$}
    \label{fig:N=3_energiesl}
\end{figure}
\begin{figure}[H]
    \centering
    \includegraphics{../img/N=3_energiesc.pdf}
    \caption{Energy for the case $N=3$, section $\chi=1$}
    \label{fig:N=3_energiesc}
\end{figure}

From equations \ref{eq:N=3_en0}, \ref{eq:N=3_en1} can be seen that $E_0=E_1$ for $D=E$, which for real values $\lambda,\;\chi$ has two solutions
$$(\lambda_d,\pm \chi_d)=\left(-\frac{1}{2};\sqrt{\frac{3}{5}}\right).$$
Point-like characteristics corresponds to \textcolor{green}{Theorem} \ref{thm:n-2}, which states that Hamiltonian driven by two real parameters can be degenerated only on 0-dimensional manifolds. 

If the energy spectrum is degenerate and metric tensor diverges, see individual elements on fig. \ref{fig:N=3_g}, its determinant also diverges, as shown on fig. \ref{fig:N=3_gDivenrgence}, along with Christoffel symbols on fig. \ref{fig:N=3_G}. Note that the metric tensor determinant is positive definite, thus the manifold is Riemannian. Further on it reflects symmetry $\chi\leftrightarrow-\chi$, except for elements $g_{12}$, $\Gamma_{121}$, $\Gamma_{211}$ and $\Gamma_{222}$, which switch their sign.
\begin{figure}[H]
    \centering
    \includegraphics[scale=1.3]{../img/N=3_gComponents.pdf}
    \includegraphics[scale=1.3]{../img/N=3_barA.pdf}
    \caption{Arcustangens of metric tensor elements for the case $N=3$. From above $Arctan(g_{11})$, $Arctan(g_{12})=Arctan(g_{21})$, $Arctan(g_{22})$}
    \label{fig:N=3_g}
\end{figure}


\begin{figure}[H]
    \centering
    \includegraphics[scale=1.3]{../img/N=3_gDivergence.pdf}
    \caption{Metric tensor determinant in a parameter space for $N=3$.}
    \label{fig:N=3_gDivenrgence}    
\end{figure}

\vspace{-40pt}
\begin{figure}[H]
    \centering
    \includegraphics[scale=1.3]{../img/N=3_gammas.pdf}    
    \includegraphics[scale=1.3]{../img/N=3_barA.pdf}
    \caption{1.0 Arcustangens of Christoffel symbols for the case $N=3$. First row from left: $Arctan(\Gamma_{111})$, $Arctan(\Gamma_{121})$, $Arctan(\Gamma_{122})$. Second row from left: $Arctan(\Gamma_{211})$, $Arctan(\Gamma_{221})$, $Arctan(\Gamma_{222})$.}
    \label{fig:N=3_G}
\end{figure}

Due to metric tensor degeneracy, the space is not \textcolor{green}{geodesically maximal}. To see that singularity is not only \emph{coordinate one}\footnote{Coordinate singularity is present only in some coordinates. This is different from so-called \emph{physical singularity}, which is present in every choice of the coordinate system.}, the Ricci scalar can be calculated, see fig. \ref{fig:N=3_Ricci}. Divergent Ricci scalar implies the existence of a \emph{physical singularity}. This can be seen from sections in $\chi$-direction drawn on fig. \ref{fig:N=3_Ricci_section}, which at coordinate $(\lambda_d;\chi_d)$ diverges, implying the singularity is \emph{physical}. 


The presence of singularities means, that our ground state manifold is \textcolor{green}{geodesically incomplete} and according to \textcolor{green}{Theorem} \ref{thm:hopf-Rinow_modified} there exist some geodesically unreachable coordinates.
\begin{figure}[H]
    \centering
    \includegraphics{../img/N=3_Ricci.pdf}
    \caption{Arcustanges of Ricci curvature for the case $N=3$. \textcolor{blue}{Excuse the resolution, I will make better calculation in Metecentrum later}}
    \label{fig:N=3_Ricci}
\end{figure}

% Coordinates of minima on lines with constant $\lambda$ in Ricci scalar can be seen on fig. \ref{fig:N=3_RicMinimas} and will be important later on, because geodesics will be strongly repelled by parts of this line.
% \begin{figure}[H]
%     \centering
%     \includegraphics[scale=0.9]{../img/N=3_Ricci_section.pdf}
%     \caption{Ricci curvature, section for three different $\lambda=const.$}
%     \label{fig:N=3_Ricci_section}
% \end{figure}
\begin{figure}[H]
    \centering
    \includegraphics{../img/N=3_RicMinimas.pdf}
    \caption{Minimas in Ricci curvature on the background of metric tensor determinant. Case $N=3$.}
    \label{fig:N=3_RicMinimas}    
\end{figure}

The system characteristics can be seen more clearly from geodesics, i.e. by solving initial value problem with conditions
$$(\lambda(t_i);\chi(t_i))=(\lambda_i;\chi_i)$$
$$\left(\der{\lambda(t)}{t};\der{\chi(t)}{t}\right)\Bigg|_{t_i}=(\lambda'_i;\chi'_i).$$

Results for these geodesics starting at $(\lambda;\chi)=(0;0)$, $(\lambda',\chi')=(\cos\theta;\sin\theta)$ for $\theta\in [-0.63;0.63]$ and $\theta\in [\pi-0.225;\pi+0.225]$ with step $0.01$, can be seen on fig. \ref{fig:N=3_geodesics}. Other values $\theta$ result in close approach of the geodesics to singularity making calculations numerically unstable. The fact that geodesics lean towards singularities is well known from the theory of General Relativity (GR). The main difference here is, that our "test particle" seems to be partially repulsed from the singularity, thus analogy with GR would fail because of nonexistence of negative mass and gravitational dipoles (at least we hope so). The better analogy would be electromagnetism, which has downside in the fact, that one is not used on working with metric structures in this theory. Comparison of those two intuitive examples can be seen on fig. \ref{fig:geodesicsinGR}. The geodesic behavior is not caused only by the singularity, but also by large Ricci curvature leaning to the right from it. The distance across this gap is thus also large, leading to the strong tendency of the geodesics to go around the singularity rather than crossing it, which is again seen in fig. \ref{fig:N=3_geodesics}.


\begin{figure}[H]
    \centering
    \includegraphics{../img/N=3_geodesics.pdf}
    \caption{Geodesics for the case $N=3$ starting from $(\lambda_i;\chi_i)=(0;0)$ with $(\lambda'_i;\chi'_i)=(\cos\theta;\sin\theta)$, parametrized by angle $\theta$.}
    \label{fig:N=3_geodesics}    
\end{figure}
\begin{figure}[H]
    \centering
    \includegraphics[width=0.6\textwidth]{../img/geodesicsinGR.png}
    \caption{Comparing geodesics with repulsing (blue) and attracting (red) metric tensor divergence in the spherical symmetrical space. }
    \label{fig:geodesicsinGR}
\end{figure}

% \begin{figure}[h]
%     \centering
%     \includegraphics{../img/N=3_geodesics_lambdaIn=-1.pdf}
%     \caption{Geodesics starting from $(\lambda_i;\chi_i)=(-1;\chi_i)$, $(\lambda',\chi')=(1;0)$, for $\chi_i\in(0;1)$ with step $0.05$. Numerically unstable geodesics were skipped. Case $N=3$.}
%     \label{fig:N=3_geodesics_lambdaIn=-1}    
% \end{figure}



\newpage
\section{Limit \texorpdfstring{$N\rightarrow \infty$}{N->infty}}
The limit $N\rightarrow \infty$ can be applied on the Hamiltonian in eq. \ref{eq:firstOrderTransitionHamiltonian} using Holstein-Primakoff mapping for bosonic operators\footnote{Felipe did this and I have no idea how}
\begin{equation}
    \mathcal{H}\coloneqq\lim_{j\rightarrow\infty}\frac{\HH}{2j},
\end{equation}
resulting in classical Hamiltonian
\begin{equation}
    \begin{split}
        \mathcal{H}(x,p)=&-\frac{1}{2}+\frac{1-\lambda}{2}x^2+\frac{\lambda-\chi^2}{4}x^4-\frac{\chi x^3}{2}\sqrt{2-x^2-p^2}-\frac{\chi^2}{4}p^4\\
        &+\frac{p^2}{4}\left[2+(\lambda-2\chi^2)x^2-2\chi x\sqrt{2-x^2-p^2}\right].
    \end{split}
    \label{eq:HamiltonianClassicalLimit}
\end{equation}


Finding minimas in its derivatives, we get the \emph{separatrix}
\begin{equation}
    \chi^2=\frac{\lambda-1}{\lambda-2},
    \label{eq:separatrix}
\end{equation}
which represents phase transition in the limit $N\rightarrow \infty$. In our case, the transition is of first order everywhere, except in $(\lambda;\chi)=(1;0)$, where it has order two. The separatrix is shown on fig. \ref{fig:transitionCompare} compared to minimum between the ground state and first excited state $E_1-E_0$ for $N=3$ case. With increasing $N$, it converges to the separatrix line.

\begin{figure}[H]
    \centering
    \includegraphics{../img/infiniteN_transitionCompare.pdf}
    \caption{First order phase transition -- Separatrix (red), second order transition (blue point) compared to minimum between the ground state and first excited state in N=3 case (black, dashed).}
    \label{fig:transitionCompare}    
\end{figure}


\section{Arbitrary $N$}
For higher dimensinos we see the same characteristic behaviour in the energy spectrum sections, see example on fig. \ref{fig:N=10_energiesl}, \ref{fig:N=10_energies2} for the $N=10$ case. Between most energy levels is at least one avoided crossing \textcolor{red}{and between zeroth and first there are $N-2$ crossings for $N$ odd and $N-3$ for $N$ even, which I have no idea how to prove, but it looks like it might be true.}


\begin{figure}[H]
    \centering
    \includegraphics{../img/N=10_energiesl.pdf}
    \caption{Energy spectrum as function of $\chi$, for $\lambda=1$ and $N=10$.}
    \label{fig:N=10_energiesl}    
\end{figure}
\begin{figure}[H]
    \centering
    \includegraphics{../img/N=10_energies2.pdf}
    \caption{Energy spectrum as a function of $\lambda$, for $\chi=1$ and $N=10$.}
    \label{fig:N=10_energies2}    
\end{figure}

Special attention was given to spectrum degeneracies between zeroth and first energy level. Their exact searching is numerically very expensive, thus first few cases, namely $N\in\{3,4,5,6,7\}$ were calculated, see Tab. \ref{tab:singularities}, and then proven up to a numerical precision for other cases $N\leq1000$.  
\begin{table}[H]
    \centering
    \begin{tabular}{c||c|c|c}
     N&$(\lambda_l;\pm\chi_l)$&$(\lambda_2;\pm\chi_2)$&$(\lambda_r;\pm\chi_r)$        \\ \hline\hline
     3&$(-\frac{1}{2};\sqrt{\frac{3}{5}}) $&                                    &        \\
     4&$(-3          ;\sqrt{\frac{4}{5}}) $& $(-\frac{1}{3};\sqrt{\frac{4}{5}})$&        \\
     5&$(-\frac{3}{2};\sqrt{\frac{5}{7}}) $& $(-\frac{1}{4};\sqrt{\frac{5}{7}})$&        \\
     6&$(-5          ;\sqrt{\frac{6}{7}}) $& $(-\frac{1}{5};\sqrt{\frac{6}{7}})$&$(-1          ;\sqrt{\frac{2}{3}}) $        \\
     7&$(-\frac{5}{2};\sqrt{\frac{7}{9}}) $& $(-\frac{1}{6};\sqrt{\frac{7}{9}})$&$(-\frac{3}{4};\sqrt{\frac{7}{11}}) $   
    \end{tabular}
    \caption{Singularities between zeroth and first energy levels for dimensions 3--7. Subsript $l(r)$ means is for most \emph{left(right)-wise} positioned coordinates in the $(\lambda,\chi)$-plot.}
    \label{tab:singularities}
    \end{table}
The observed formula for $(\lambda_l,\chi_l)$ and $(\lambda_r,\chi_r)$, i.e. those with minimal, resp. maximal $\lambda$ coordinate, is\footnote{notice the symmetry $\chi\leftrightarrow -\chi$}
\begin{equation}
    (\lambda_l ;\pm\chi_l)= \begin{cases}
        \left(1-\frac{N}{2};\sqrt{\frac{N}{N+2}}\right) & ,N\text{ is odd}, N\geq 3\\
        \left(1-N;\sqrt{\frac{N}{N+1}}\right) & ,N\text{ is even}, N\geq 3
    \end{cases}
    \label{eq:singularityCoordinateFormulaLeft}
\end{equation}
\begin{equation}
    (\lambda_r ;\pm\chi_r)= 
        \left(\frac{1}{1-N};\sqrt{\frac{N}{2N-1}}\right)\quad , N\geq 3.
        \label{eq:singularityCoordinateFormulaRight}
\end{equation}
Dimensions 3 to 10 are shown in fig. \ref{fig:singularities3to10}.
In addition, the degeneracies between zeroth and first energy level belong to the separatrix described by eq. \ref{eq:separatrix}. Due to this, the position of singularities is constrained on separatrix between points $(\lambda_l,\pm\chi_l)$ and $(\lambda_r,\pm\chi_r)$.



In the limit $N\rightarrow\infty$ they converge to
\begin{align*}
    \lim_{N\rightarrow \infty}(\lambda_l ;\pm\chi_l)&= \left(-\infty,1\right)\\
    \lim_{N\rightarrow \infty}(\lambda_r ;\pm\chi_r)&= \left(0,\frac{1}{\sqrt{2}}\right)
\end{align*}
and because there is an infinite number of singularities and because the whole separatrix is divergent, they need to be dense.


Few characteristics in arbitrary dimension can be shown on the case $N=5$. First are degeneracies between different energy levels, which can be seen on fig. \ref{fig:singularitiesBetweenEnergiesN=6}. One can see, that only $E_0=E_1$ degeneracy lies on the separatrix and all others are distributed with $\chi\leftrightarrow-\chi$ symmetry. From $E_2-E_1$ plot, the separatrix is also apparent as local maxima between two local minima strips (darker parts of the plot), but in higher energy levels, it is much harder to see it.
 



% \begin{figure}[h]
%     \centering
%     \includegraphics{../img/divPosition.pdf}
%     \caption{Singularities in the metric tensor for of $N\in\{3,4,\dots ,14\}$ and separatrix according to eq. \ref{eq:separatrix}.}
%     \label{fig:singularitiesRegr}    
% \end{figure}

\begin{figure}[h]
    \centering
    \includegraphics[scale=1.3]{../img/singularitiesPlots.pdf}
    \caption{Spectrum degeneracy between $E_0$ and $E_1$. Hamiltonian dimensions are 1,3,5,7 in the first column and 2,4,6,8 in the second column. Black crosses mark most left-wise and right-wise singularity and the background corresponds to the metric tensor determinant. Other singularities are also well visible in the determinant.}
    \label{fig:singularities3to10}    
\end{figure}



\begin{figure}[h]
    \centering
    \includegraphics[scale=1.3]{../img/singularitiesBetweenEnergiesN=6.pdf}
    \includegraphics[scale=1.3]{../img/N=5_bar6.pdf}
    \caption{Energy differences between neighboring layers for $N=5$. From top left to the right left: $E_1-E_0$, $E_2-E_1$, $E_3-E_2$, $E_4-E_3$, $E_5-E_4$, $E_6-E_5$. \textcolor{red}{Not all degeneracies are shown, due to faulty numerics, but probably all black points are degeneracies.} }
    \label{fig:singularitiesBetweenEnergiesN=6}    
\end{figure}



