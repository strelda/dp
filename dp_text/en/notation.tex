
\chapwithtoc{Some notes to the notation}

\begin{tabular} {@{}C{1.9cm}@{}p{8cm}@{}C{3.77cm}}
	\toprule
	\textbf{Symbol}& \textbf{Meaning}& \textbf{Defining formula}\\\bottomrule
	$\A$ & Gauge (calibrational) potential & $\A_\mu=i\hbar \partial_\mu$ \\
	$\mathbb{N}$ & Natural numbers, without zero \\
	$\mathcal{C}^k$ & k-times differentiable function \\
	
\bottomrule
\multicolumn{3}{l}{\footnotesize}
\end{tabular}

Mathematical spaces will be denoted \emph{mathcal} and operators with \emph{hat}.

The colored text will be sometimes used during the derivations. The text can be understood without the colors, its goal is strictly pedagogical and helps reader to see some underlying connections.