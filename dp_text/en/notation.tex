
\chapwithtoc{Some notes to the notation}

\vspace{-0pt}\begin{tabular} {@{}C{1.9cm}@{}p{8cm}@{}C{3.77cm}}
	\toprule
	\textbf{Symbol}& \textbf{Meaning}& \hspace{-30pt}\textbf{Characterizing formula}\\\bottomrule
	$\A_\mu$ & Calibrational (gauge) potential & $\A_\mu\coloneqq i\hbar \partial_\mu$ \\
	$A_\mu$ & Berry connection & $A_\mu\coloneqq \braket{o|\A_\mu|o}$ \\
	$\mathcal C$ & closed driving curve in parametric space &  \\
	$\mathcal{C}^k$ & k-times continuously differentiable function \\
	$\C$ & complex numbers & \\
	  $\gamma_s$& geometrical phase induced by $s^{\text{th}}$ energy level &  \\
	  $D_\mu$&covariant derivative  &  \\
	  $E_s$&$s^{\text{th}}$ energy level of Hamiltonian &  \\
	  $F$& fidelity & \hspace{-7pt}$F(\ket{\psi},\ket{\phi})=|\!\braket{\psi|\phi}\!|^2$  \\
	  $F^*$& infidelity & $F^*\coloneqq 1-F$  \\
	  $f$& fidelity amplitude  & $|f|^2\coloneqq F$  \\
	  $g_{\mu\nu}$& metric tensor& $g_{\mu\nu}= \Re\chi_{\mu\nu}$ \\
	  $\HH(\llambda)$& parameter controlled Hamiltonian  &  \\ 
	  $\mathcal H$& Hilbert space  &  \\ 
	  $\mathcal H(\llambda)$& Hilbert space for $\HH(\llambda)$ &  \\ 
	  $\mathrm{Ind}_a$ & winding number around point $a$ & \\
	  $\bm\J$ & angular momentum operator & \\
	  $\mathcal J$ & driving curve in parametric space &  \\
	  $\llambda$& $n$-dimensional real parameter of Hamiltonian &$\llambda\in \mathcal U\subset \R^n$  \\
	  $\M$& manifold &  \\
	  $\M_s$& $s^{\text{th}}$ energy state manifold & $\hspace{-43pt}\M_s\coloneqq\cup_{\llambda\in\mathcal U,\;\varphi\in[0,2\pi)} e^{i\varphi}\ket{s(\llambda)}$ \\
	  $N$ & dimension of the Hamiltonian & \\
	  $\mathbb{N}$ & natural numbers (excluding zero) \\
	  $\ket{o(\llambda)}$& ground state of Hamiltonian $\HH(\llambda)$  &  \\
	  $\mathcal O$& order of the function (Big O notation)  &  \\
	  $\P\M_s$& $s^{\text{th}}$ energy projective state manifold & $\P\M_s\coloneqq\cup_{\llambda\in\mathcal U} \ket{s(\llambda)}$ \\
	  $\varphi_B$ & Berry phase & $\varphi_B\coloneqq-\oint_\mathcal{C} A_j(\llambda)\d \llambda^j$\\
	  $\R$& Real numbers  &  \\
	  $R^\alpha_{\;\;\mu\nu\kappa}$ &Riemann tensor & \\
	  $R_{\mu\nu}$ & Ricci tensor & $R_{\mu\nu}\coloneqq R^\alpha_{\;\;\mu\alpha\nu}$\\
	  $R\equiv \mathrm{Ric}$ & Ricci scalar, or Ricci curvature & $R\coloneqq R^\mu_{\;\;\mu}$\\
	  $\nu_{\mu\nu}$ & Berry curvature & $\nu_{\mu\nu}= \Im\chi_{\mu\nu}$ \\ 
	  $\ket{s}$& eigenstate corresponding to energy $E_s$& $\HH\ket{s}=E_s\ket{s}$ \\
	  $t$& time  &  \\
	  $T$& final time of driving  &  \\
	  $\mathbb T_a \M$& tangent field of $\M$ in $a\in \M$ &  \\
	  $\mathbb T_a^* \M$& cotangent field of $\M$ in $a\in \M$ &  \\
	  $\T^p_q\M$& $q$-times covariant and $p$-times contravariant tensor field of $\M$&  \\
	  $\mathcal U$& parametric space, open subset of $\R^n$  & \\
	  $U(k)$& $k$-parametric unitary transformation group  &  \\
	  $\chi_{\mu\nu}$ & geometric tensor & $\chi_{jk}\coloneqq \braket{\partial_j o|\partial_k o}_c,
	  $ \\
	  $(\chi;\lambda)$& LMG model parameters  & $(\chi,\lambda)\equiv \llambda\in\R^2$ \\
\bottomrule
\multicolumn{3}{l}{\footnotesize}
\end{tabular}


	  \begin{tabular} {@{}C{4cm}@{}p{9.67cm}}
	\toprule
	\textbf{Operation}& \textbf{Notation}\\\bottomrule	
	commutator & $[A,B]$ \\
	connected braket & ${\partial_j o|\partial_k o}_c \equiv \braket{\partial_j o|\partial_k o} - \braket{\partial_j o|o}\braket{o|\partial_k o}$\\
	fiber space & $(\mathcal{E},\mathcal{U},\pi,\mathcal{F})\equiv(\text{total space},\text{base space}, \text{surjection},\text{fiber})$\\
	Poisson braket & $\{A,B\}$ \\
	quantum braket & $\braket{\psi|\phi}$\\
	\bottomrule
\end{tabular}


\begin{tabular} {@{}C{4cm}@{}p{9.67cm}}
	\toprule
	\textbf{Shortcut}& \textbf{Full name}\\\bottomrule
	
	APT & adiabatic perturbation theory \\
	LZS & Landau-Zener-Stueckelberg\\
	LMG & Lipkin-Meshkov-Glick \\
	\bottomrule
\end{tabular}


\begin{itemize}
    \item Quantum operators are denoted with \emph{hat}, 
    \item abstract indices are written in \emph{Greek}, pointer indices in \emph{Latin}. 
    \item If the object is defined by the formula on right, “$\coloneqq$” is used. If the equality holds by definition, “$\equiv$” is used. 
    \item Coordinate derivative can be denoted using a comma before index and covariant derivative using a semicolon. 
    \item The colored text is sometimes used. The text can be understood without the colors, its goal is strictly pedagogical and helps the reader to see some underlying connections.
\end{itemize}