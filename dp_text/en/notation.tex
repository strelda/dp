
\chapwithtoc{Some notes to the notation}

\begin{tabular} {@{}C{1.9cm}@{}p{8cm}@{}C{3.77cm}}
	\toprule
	\textbf{Symbol}& \textbf{Meaning}& \textbf{Defining formula}\\\bottomrule
	$\A$ & Gauge (calibrational) potential & $\A_\mu=i\hbar \partial_\mu$ \\
	$\mathbb{N}$ & Natural numbers, without zero \\
	$\mathcal{C}^k$ & k-times differentiable function \\
	
\bottomrule
\multicolumn{3}{l}{\footnotesize}
\end{tabular}

Few more notes to notation:
\begin{enumerate}
	\item Quantum operators are denoted with \emph{hat}. Coordinate derivative is sometimes denoted using comma and covariant derivative using semicolon. 
	\item Abstract indices are written in \emph{greek}, pointer indices in \emph{latin}. 
	\item If the object is defined by the formula on right, "$\coloneqq$" is used. If the equality holds by definition, "$\equiv$" is used. 
	\item The colored text is sometimes used during the derivations. The text can be understood without the colors, its goal is strictly pedagogical and helps reader to see some underlying connections.
\end{enumerate}