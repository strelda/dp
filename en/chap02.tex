\chapter{Mathematical introduction}
The modern approach to the closed system dynamics is using differential geometry formalism. Before we get to the quantum mechanics itself, lets breathly define the formalism recapitulate some definitions of this part of mathematics. More detailed and structured notes can be found for example in \citep{fecko}.

Let's have a manifold $\M$ and curves 
$$\gamma:\R \overset{open}{\supset} I \rightarrow \M \qquad \xi\mapsto \gamma(\xi).$$ 
The space of functions is $\F(\M)\equiv\{f:\M\rightarrow \R\}$, where 
$$f:\M\rightarrow U\overset{open}{\subset} \R \qquad x\mapsto f(x).$$
To define \emph{vectors} on $\M$, we need to make sense of the \emph{direction}. It is defined using curves satisfying 
$$\gamma_1(0)=\gamma_2(0)\equiv P$$
$$\der{}{t}x^i(\gamma_1(t))\big|_{t=0}=\der{}{t}x^i(\gamma_2(t))\big|_{t=0}.$$
Taking the equivalence class created by those two rules, sometimes noted as $[\gamma]=v$, we have element of the tangent space to $\M$. We will use standard notation for the tangent space to $\M$ in some point $xP$ as $\T_P\M$ and contangent space as $\T^*_P\M$. Unifying all those spaces over all $x$ we get tangent and cotangent bundle, $\TT\M$ and $\TT^*\M$ respective. To generalize this notation to higher tensors, we denote $\T_P\M\in \TT^1\M$, $\T^*_P\M\in \TT_1\M$, thus the space of $p-$times contravariant and $q-$times covariant tensors is denoted $\TT^p_q\M$.

Using the congruence of the curves on $\M$, the expression 
\begin{equation}
    \der{}{\xi}f\circ \gamma(\xi)\Big|_{\xi=0}
\end{equation}
has a good meaning and we can define the \emph{derivative} in some $P\in\M$ as
\begin{equation}
    \bm v: \F(\M)\rightarrow \R \qquad f\mapsto \bm v[f]\equiv \frac{\d f(\gamma(\xi))}{\d \xi}\Big|_P \equiv \partial_\xi\Big|_P f .
\end{equation}
It holds, that $\bm v\in \T_P\M$ and can be expressed as the \emph{derivative in direction},\footnote{
        The direction itself is usually denoted as
        \begin{equation}
            \frac{\D}{\d\bm\alpha}\gamma(\xi),
        \end{equation}
        where the big D notation is used to point out that it's not a classical derivative, but it maps curves to some entirely new space of directions.
    } 
which can be understood in coordinates as
\begin{equation}
    \bm v[f] = \der{}{\bm v} f\circ \gamma(\xi)\Big|_{\xi=0}=v^\mu\der{}{x^\mu} f(\bm x)\Big|_{P}.
\end{equation}
The directionnal derivative will be denoted 
$$\nabla_v$$
and in basis $\bm e_i \equiv \partial/\partial x^i$ we will denote 
$$\nabla=(\bm e_x, \bm e_y,\bm e_z)$$



To get some physical application, we need to define one strong structure on manifolds -- differentiable metric tensor $g_{\mu\nu}\in\TT^0_2\M$ -- so the covariant derivatives and parallel transport are well-defined everywhere. 


\section{Pull-back and push forward}
Push-forward and pull-back are used to transport vectors and covectors between manifolds. Let's have two manifolds $\M$, $\mathcal{N}$, a smooth mapping $\phi$ and functions $f,\tilde f$ such that
\begin{align*}
    \phi&:\M\rightarrow \mathcal{N}\qquad x\mapsto \phi x\\
    \tilde f&:\mathcal{N}\rightarrow \R 
\end{align*}
\emph{Pull-back of the function} then defines a new function $
f:\M\rightarrow \R $ as
$$\phi^*:\F \mathcal{N}\rightarrow \F\M \qquad  \tilde f\mapsto f=(\phi^*\tilde f)(x)\equiv \phi^*\tilde f(x) =\tilde f(\phi x).$$
\emph{Push-forward of a vector} is defined as
$$\phi_*: \T_x\M \rightarrow \T_{\phi x}\mathcal N\qquad \phi_* 
\Der{\gamma(\xi)}{\xi}\Big|_x=\Der{\phi \gamma(\xi)}{\xi}\Big|_x$$
and \emph{pull-back of a covector} $\bm\tilde\alpha\in \T_{\phi x}\mathcal N$ is
$$\phi^*: \T_{\phi x}\mathcal N\rightarrow \T_x\M  \qquad (\phi^*\bm \tilde\alpha)_\mu v^\mu\big|_x= \tilde\alpha_\mu (\phi_* \bm v)^\mu\big|_{\phi x}.$$
If $\phi$ has a smooth inversion, i.e. it is a dippheomorphism, we can define pull-back of vectors as
\begin{equation}
    \phi^*=\phi_*^{-1}
\end{equation}
and push-forward of covectors
\begin{equation}
    \phi_*=(\phi^{-1})^*
\end{equation}

\section{Covariant derivative and parallel transport}
Covariant derivative is generally\dots
Metric covariant derivative is\dots


Affine connection can be expressed as
\begin{equation}
    \Gamma^{\alpha}_{\mu\nu} = \frac{1}{2}g^{\alpha \beta}\left(g_{\beta\mu,\nu}+g_{\nu\beta,\mu}-g_{\mu\nu,\beta}\right),
\end{equation}
where we used comma notation for the coordinate derivative.
The covariant derivative of $\bm a\in \T_P\M$ is then defined
\begin{equation}
    \Der{a^\mu}{x^\nu}=a_{\;,\nu}^\mu-\Gamma^\mu_{\alpha\beta} x^\alpha a^\beta 
\end{equation}
and for $\bm\alpha\in \T_P^*\M$ it is
\begin{equation}
    \Der{\alpha_\mu}{x^\nu}=\alpha_{\mu,\nu}-\Gamma^\alpha_{\mu\beta}x^\beta \alpha_\alpha 
\end{equation}

The vector $v\in \T_P\M$ is said to be parallel transported along curve $\gamma(\lambda)$, if it's covariant derivative
\begin{equation}
    \Der{v^\mu}{\xi}=0
\end{equation}
vanishes along $\gamma$.

\section{Antisymmetric tensors and wedge product}
p-form $A\in \TT_p \M$ is called \emph{antisymmetric}, if changing the order of the indices has impact only on the sign, symbolically
$$A_{i_1\dots i_p} = \sign(\sigma)A_{i_{\sigma_1}\dots i_{\sigma_p}},$$
where $\sigma$ is some permutation.\emph{Antisymmetrisation} is defined as a normalized sum over all permutation
\begin{equation}
    A^{[i_1\dots i_p]}\equiv \frac{1}{p!}\sum_\sigma A^{[i_{\sigma_1}\dots i_{\sigma_p}]}. 
\end{equation}

The \emph{wedge product} of $A\in \TT_p \M$ and $B\in \TT_q \M$ is antisymmetrisation of the tensor product in the sense
\begin{equation}
    A\wedge B\equiv \frac{(p+q)!}{p!q!} A^{[i_1\dots i_p}\otimes B^{i_1\dots i_q]}
\end{equation}



\chapter{Physical introduction}
Most parts of this chapter are inspired by \citep{kolodrubez} and original notes \citep{berry1984}, \citep{berry1989}, \citep{berry2009}
Now we will assign some physical background to the structure defined in the first chapter.

Assume manifold $\M$ generated by eigenstates of some closed system Hamiltonian $\HH(\llambda)$, meaning the Hamiltonian is bounded and dimension of the space is finite. Let's assume the existence of $\mathcal{C}^1$ mapping\footnote{is continuous to the first derivative} (parametrisation) $B: \M\rightarrow \bm\llambda\equiv (\lambda^1,\dots,\lambda^n)\in\R^n$. Therefore, we will denote eigenstates in any instant $\ket{\iota(\bm\llambda)}$ and their energies $E(\bm\llambda)$. This means, that they depend on time, iff Hamiltonian depends on time and because the Hamiltonian is assumed to be finitedimensional, they are discrete.


\citep{berry1984} The state of the system evolves according to \Schrodinger equation
\begin{equation}
    i\hbar \d_t\kpsit = \HH(\llambda(t))\kpsit,
    \label{eq:schrodinger}
\end{equation}
which for eigenstates reads as
\begin{equation}
    \HH(\llambda(t))\ket{n(\llambda(t))}=E_n(\llambda(t))\ket{n(\llambda(t))}
    \label{eq:energySchrodinger}
\end{equation}
with solution
\begin{equation}
    \kpsit = \exp\left(-\frac{i}{\hbar}\int_0^tE_n(\llambda(\tau)\d\tau)\right)\exp(i\gamma_n(t))\ket{n(\llambda(t))}.
\end{equation}
The first exponential, the \emph{dynamical phase}, is well known, if we switch of the independence on $\llambda$. The second exponential is called \emph{geometrical phase} and yields, that some topology on $\M$ must be induced. It is generally non-integrable, meaning it cannot be written simply as $\gamma_n(\llambda(t))$ and for some closed curve on $\M$
\begin{equation}
    C=\{\llambda(t)|t\in[0,T] \text{, such that }\llambda(0)=\llambda(T)\}
\end{equation} 
we generally get $\gamma(0)\neq \gamma(T)$.

Substituting this general solution to eq. \ref{eq:schrodinger}, we get
\begin{equation}
    \d_t \gamma(t)=i\braket{n(\llambda(t))|\nabla_\llambda n(\llambda(t))}\cdot \d_t \llambda(t).
\end{equation}
Integrating this equation around some closed curve $C$ and assuming the dynamical phase to be zero, thus not exciting the system, we get
\begin{equation}
    \gamma_n(C)=i\oint_C\braket{n(\llambda)|\nabla_\llambda n(\llambda)}\cdot \d \llambda.
    \label{eq:gammaCoint}
\end{equation}
We see, that the geometric phase does not depend on the direction if integration, only on the path itself. The problem of this expression lies in $\partial_\llambda n(\llambda)$, which locally requires single-valued basis $\{\ket{n}\}_n$. This can be avoided in 3-dimensions using Stokes's theorem for $S$ as the surface with boundary $\partial S=C$.
\begin{equation}
    \begin{split}
        \gamma_n(C) &= -\Im \iint_C \d S \cdot \nabla \times \braket{n(\llambda)|\nabla n(\llambda)}\\
         &= -\Im \iint_C \d S \cdot \braket{\nabla n(\llambda)|\times|\nabla n(\llambda)}\\
        &= -\Im \iint_C \d S \cdot \sum_{m\neq n} \braket{\nabla n(\llambda)|m(\llambda)}\times \braket{m(\llambda)|\nabla n(\llambda)}\\
        &= -\iint_C \d S \cdot V_n(\llambda)
            \label{eq:stokes}
    \end{split}
\end{equation}
for 
\begin{equation}
    V_n(\llambda) = \Im \frac{
            \braket{n(\llambda)\nabla_\llambda \HH(\llambda) |m(\llambda)}\times \braket{m(\llambda)|\nabla_\llambda \HH(\llambda)|n(\llambda)}    
             }{
(E_m(\llambda)-E_n(\llambda))^2
            }
\end{equation}
where the element of summation $m=n$ is real, therefore has no influence on $\gamma_n$ and can be omitted. The last equivalence holds, because
\begin{equation}
    \braket{m(\llambda)|\nabla n(\llambda)}=
    \frac{\braket{m(\llambda)|\nabla \HH |n(\llambda)}}
    { E_n(\llambda)-E_m(\llambda)}, \qquad n\neq m.
\end{equation}
Also comparing the first expression in eq. \ref{eq:stokes} with it's last one we see, that
\begin{equation}
    V_n(\llambda)=\nabla\times\braket{n(\llambda)|\nabla m(\llambda)}, 
    \label{eq:vectorPotentialDef}  
\end{equation}
defining its vector potential. In addition, it extends our definition from single values basis to any solution of \ref{eq:energySchrodinger}.

As was mentioned, the above procedure from eq. \ref{eq:gammaCoint} was performed only for three dimensional space. Proper generalization to n-dimensional space would yield, see \citep{berry1984},
\begin{equation}
    \gamma_n(C) = -\iint_C \d S \cdot\Im \frac{
            \braket{n(\llambda)\d \HH(\llambda) |m(\llambda)}\wedge \braket{m(\llambda)|\d\HH(\llambda)|n(\llambda)}    
             }{
(E_m(\llambda)-E_n(\llambda))^2
             }.
\end{equation}
From the fact, that this does not depend on energy of the system, but only on the geometry of the \textcolor{red}{curve on manifold, i.e. on sequence of Hamiltonians along the path and not on it's time history.}


\begin{definition}[Anholonomy]
    Geometrical phenomenon, which causes some variable $V(\gamma(p))$ not to return to it's original value while varying it's parameter $p$ around some closed curve $\gamma(p)$. 
\end{definition}
If the transporting variable is quantum state, we can measure some non-zero angle between $\ket{V(\gamma(0))}$ and $\ket{V(\gamma(p))}$, whilst $\gamma(p)=\gamma(0)$, meaning
$$\braket{V(\gamma(0))|V(\gamma(p))}\neq 0.$$

\begin{definition}[Adibaticity]
    Slow change in a sense, that it does not excite the system and allows the system to return to the same energetic state after circulation around any closed path on the manifold. For more see Theorem \ref{adiabaticTheorem}.
\end{definition}




\section{Metric and geometric tensor}
Now we need to find some reasonable way to measure the distance on $\M$. Our first guess might be
\begin{equation}
    \d \tilde{s}^2 = \braket{\iota(\bm\llambda+\d\bm\llambda)|\iota(\bm\llambda+\d\bm\llambda)} = 1-2\Re{\braket{\iota(\bm\llambda+\bm\d\llambda)|\iota(\bm\llambda)}}.
\end{equation}
This is \emph{gauge dependent}, meaning that it depends on our choice of the wave phase, i.e. on observer. The phase change $\phi$ induces the change as $\ket{n(\llambda)}\rightarrow e^{i\phi(\llambda)} \ket{n(\llambda)}$, leading to $\braket{n(\llambda)|\nabla n(\llambda)}\rightarrow \braket{n(\llambda)|\nabla n(\llambda)} + i\nabla \phi(\llambda)$. For $\phi(\llambda)\in \mathcal C^2$ we see from eq. \ref{eq:vectorPotentialDef}, that gauge independent choice would be for example 
\begin{equation}
    f=\braket{\iota(\bm\llambda+\d\bm\llambda)|\iota(\bm\llambda)},
\end{equation}
sometimes refered to as the \emph{fidelity}. We can see it's physical meaning imagining \emph{quantum quench} (rapid change of some Hamiltonian parameters), in which case $f^2$ is the probability that system will remain in the new ground state. $1-f^2$ is therefore probability to excite the system during this quench, which leads to the definition of \emph{distance on $\M$}
\begin{equation}
    \d s \equiv 1-f^2= 1-\left|\braket{\iota(\bm\llambda+\d\bm\llambda)|\iota(\bm\llambda)}\right|.
\end{equation}
Using $\d s^2 = g_{\mu\nu}\d \llambda^\mu \d\llambda^\nu+\O(\llambda^3)$, we get the metric tensor
\begin{equation}
    g_{\mu\nu}^{(i)}(\bm\llambda) = \Re\left(\braket{\partial_{\lambda^\mu}\iota(\bm\llambda)|\partial_{\lambda^\nu}\iota(\bm\llambda)} - \braket{\partial_{\lambda^\mu}\iota(\bm\llambda)|\iota(\bm\llambda)}\braket{\iota(\bm\llambda)|\partial_{\lambda^\nu}\iota(\bm\llambda)}\right).
    \label{eq:geom.tensorDefinition}
\end{equation}

Let's have a initial state described by Hamiltonian $\H_\iota=\H(\llambda)$ in eigenstate $\ket{\iota(\llambda)}$, which undergoes the change of parameters $\llambda\rightarrow \llambda+\d \llambda$ resulting in the Hamiltonian $\H_f$ with eigenstates $\ket{\psi_n(\llambda+\d \llambda)}$, $n\in \{1,\dots,dim(\H_f)\}$. Probability amplitude of going to some specific excited state is
\begin{equation}
    \begin{split}
        a_n&=\braket{\psi_n(\llambda+\d\llambda)|\iota(\llambda)}\approx \d\lambda^\mu\braket{\partial_\mu \psi_n(\llambda)|\iota(\llambda} \\
        &= -\d\lambda^\mu\braket{\psi_n(\llambda)|\partial_\mu|\iota(\llambda)}\equiv -\d\lambda^\mu\braket{n|\partial_\mu|\iota},
    \end{split}
\end{equation}
where we introduced shortend notation for eigenstates of the Hamiltonian $\H_0$. If we introduce the \emph{gauge potential}, a.k.a \emph{calibration potential} as
\begin{equation}
    \AA_\mu\equiv i\hbar \partial_{\mu}
\end{equation}
and rescale units to $\hbar=1$, as we will use further on, we get
\begin{equation}
   a_n=\sum_\mu i\braket{n|\AA_\mu |\iota}\d\lambda^\mu,
\end{equation}
which has meaning of matrix elements of the gauge potential. Probability of the excitation i.e. transition to any state $n>0$ is then
\begin{equation}
    \begin{split}
        \sum_{n\neq 0}|a_n|^2&=  \sum_{n\neq 0} \d \lambda^\mu \d \lambda^\nu\braket{\iota|\AA_\mu|n}\braket{n|\AA_\nu|\iota}+\O(|\d \lambda^3|) = \d \lambda^\mu \d \lambda^\nu\braket{\iota|\AA_\mu \AA_\nu|\iota}_c\\
        &= \d \lambda^\mu \d \lambda^\nu\chi_{\mu\nu}+\O(|\d \lambda^3|)=\d s^2+\O(|\d \lambda^3|),
    \end{split}
\end{equation}
where we defined \emph{connected correlation function}, or \emph{covariance}
\begin{equation}
    \braket{\iota|\AA_\mu\AA_\nu|\iota}_c\equiv \braket{\iota|\AA_\mu\AA_\nu|\iota} - \braket{\iota|\AA_\mu|\iota}\braket{\iota|\AA_\nu|\iota}.
    \label{eq:covariance}
\end{equation}
If we leave out $\hbar$, we have the \emph{geometric tensor}\footnote{sometimes defined directly as the expression in eq. \ref{eq:covariance}}
\begin{equation}
    \chi_{\mu\nu}\equiv \braket{\partial_\mu \iota|\partial_\nu \iota}_c = \braket{\partial_\mu \iota|\partial_\nu \iota} - \braket{\partial_\mu \iota|\iota}\braket{\iota|\partial_\nu \iota},
\end{equation}
where $\ket{\partial_\nu \iota}\equiv\partial_\nu \ket{ \iota}$. Because $\chi$ is Hermitian ($\chi_{\mu\nu}=\chi^*_{\nu\mu}$), only the symmetric part adds up to the distance between states 
\begin{equation}
    \d s^2= g_{\mu\nu}\d \lambda^\mu \lambda^\nu= \chi_{\mu\nu}\d \lambda^\mu \lambda^\nu.
\end{equation}
 and only the symmetric part determines the distance between the states. Therefore it's practical to decompose it as
\begin{equation}
    \chi_{\mu\nu} \equiv g_{\mu\nu} - \textcolor{purple}{i\frac{1}{2}} \nu_{\mu\nu},
\end{equation}
where the \emph{Fubini-Study tensor}, as it's called, is
\begin{equation}
    g_{\mu\nu} = \frac{\chi_{\mu\nu}+\chi_{\nu\mu}}{2} = \Re\braket{\partial_\mu i|\partial_\nu i}_c = \textcolor{purple}{\Re \sum_{i\neq j}\frac{\braket{\iota|\pder{\H}{\lambda^\mu}|j}\braket{j|\pder{\H}{\lambda^\nu}|\iota}}{(E_i-E_j)^2}},
    \label{eq:geom.tensorREdefinition}
\end{equation}
and the \emph{curvature tensor} a.k.a. \emph{Berry curvature} is
\begin{equation}
    \begin{split}
        \nu_{\mu\nu} = 2 i(\chi_{\mu\nu}-\chi_{\nu\mu})&= \Im\braket{\iota|[\overleftarrow{\partial}_\nu,\partial_\mu]|\iota}_c = -2 \Im \sum_{i\neq j}\frac{\braket{\iota|\pder{\H}{\lambda^\mu}|j}\braket{j|\pder{\H}{\lambda^\nu}|\iota}}{(E_i-E_j)^2},
    \end{split}
\end{equation}
where $\overleftarrow{\partial}_\nu$ is the derivative of the covector on the left.
Because $g_{\mu\nu}$ is positive semidefinite, it really can be used as the metric tensor.

Fubini-Study tensor can be seen as the Pull-back of the elements of the full Hilbert space to $\M$. 


Next we define the \emph{Berry connection}
\begin{equation}
    A_\mu\equiv \braket{\iota|\AA_\mu|\iota},
\end{equation}
which empovers us to write
\begin{equation}
    \nu_{\mu\nu} = \partial_\mu A_\nu-\partial_\nu A_\mu
\end{equation}
and \emph{Berry phase}
    \footnote{
        The reasonability of this definition can be seen, if we assume the ground state of a free particle
            $\braket{\bm{x}|\iota}=\iota(\bm{x},\llambda)= |\iota(\bm{x})|e^{i\phi(\llambda)}$,
        then the Berry connection is
        \begin{equation}
            A_\mu=-\int \d \bm{x}|\iota|^2\partial_\mu \phi = -\partial_\mu \phi
        \end{equation} 
        and Berry phase
        \begin{equation}
            \phi_B=\oint_\mathcal{C} \partial_\mu \phi \d \lambda^\mu,
        \end{equation}
        which represents total phase accumulated by the wavefunction. It is really the analogy for Berry phase in classical mechanics, which for example for the Faucolt pendulum on one trip around the Sun makes $\phi_B=2\pi$
    }
\begin{equation}
    \phi_B\equiv-\oint_\mathcal{C} A_\mu \d \lambda^\mu=\int_\mathcal{S} F_{\mu\nu}\d \lambda^\mu \wedge \d\lambda^nu,
\end{equation}
where we used the Stokes theorem defining, that the curve $\mathcal{C}$ surrounds some area $\mathcal{S}$.

Wave-functions are elements of the tangent bundle $\TT\in \M$, the gauge potentials are affine connections defining the parallel transport. Covariant derivative is
\begin{equation}
    D_\mu=\partial_\mu+\frac{i}{\hbar}\AA_\mu,
\end{equation}
which yields $D_\mu\ket{\psi_n}=0$ for every eigenstate, which encloses the circle and \textcolor{red}{justifies our initial choise for the distance on $\M$.}








\section{Gauge potentials}
Adiabatic transformation is such a transformation from $\M$ to $\M$, which does not excite the system. Generally it can be achieved by two ways -- infinitely slow transformation of states, or adding some \emph{counterdiabatic elements} to the Hamiltonian to counter the excitation.


In case of adiabatic gauge potential we choose the basis for $\M$ as eigenstates of the Hamiltonian of the full system $\H$. Adiabatic transformation can be understood as parallel transport and adiabatic potentials as affine connection. To understand it more, let's first consider classical system and then move to the quantum mechanics.


\textcolor{blue}{\emph{move elsewhere}: In the case of simple systems, the adiabatic potentials can be found analytically, but for more complicated Hamiltonians we will be forced to use approximations, or some perturbational and variational methods.}




\section{Classical gauge potential}
In the Hamiltonian classical mechanics, we assume the manifold $\M$ \textcolor{red}{to be an accessible part of the phase space} using the Hamiltonian $\H=\H(p_i,q_i)$, where momentum $p_i$ and position $q_i$ are assumed to form the orthogonal basis of the phase space, i.e.
\begin{equation}
    \{q^i,p_j\}=\delta^i_j,
    \label{eq:canonicalCommutationDelta}
\end{equation}
which also defines \emph{calibrational freedom} in their choice. \emph{Canonical transformations} then by definition preserve this formula. Using the \emph{Poisson bracket}, defined as
\begin{equation}
    \{A,B\}\equiv \pder{A}{q^j}\pder{B}{p_j}-\pder{B}{q^j}\pder{A}{p_j},
\end{equation}
we will examine continuous canonical transformations generated by gauge potential $\A_\lambda$
\begin{align}
        q^j(\lambda+\delta\lambda)&=q^j(\lambda)-\pder{\A_\lambda,\bm{p},\bm{q}}{p_j}\delta\lambda \;\Rightarrow\; \pder{q^j}{\lambda}=-\pder{\A_\lambda}{p_j}=\{\A_\lambda,q^j\}
        \label{eq:gaugeAsGeneratorOfMotion1}\\
        p_j(\lambda+\delta\lambda)&=p_j(\lambda)-\pder{\A_\lambda,\bm{p},\bm{q}}{q^j}\delta\lambda \;\Rightarrow\; \pder{p_j}{\lambda}=-\pder{\A_\lambda}{q^j}=\{\A_\lambda,p_j\}.
        \label{eq:gaugeAsGeneratorOfMotion2}
\end{align}
Substituting this to eq. \ref{eq:canonicalCommutationDelta}, we get
\begin{equation}
    \{q^j(\lambda+\delta\lambda),p_j(\lambda+\delta\lambda)\}=\delta^i_j + \mathcal{O}(\delta\lambda^2).
\end{equation}
 
Equations \ref{eq:gaugeAsGeneratorOfMotion1},\ref{eq:gaugeAsGeneratorOfMotion2} are identical to the Hamilton equations
\begin{equation}
\begin{split}
    \dot{q}^j&=-\{\H,q^j\} = \pder{\H}{p_j}\\
    \dot{p}_j&=-\{\H,p_j\} = -\pder{\H}{q^j},
\end{split}
\end{equation}
if $\A_t=-\H$. Because the Hamiltonian is generator of the movement in the phase space $(\bm{q},\bm{p})$, we can interpret $\A_t$ as the generators of the movement on $\M$. Specially if we chose $\lambda=X^i$, we get $\A_{X^i}=p_i$.




\section{Quantum gauge potential}
\citep{kolodrubez}[kap. 2.2]
The role of Poison brackets in quantum mechanics is taken by commutators, canonical transformations are called \emph{unitar transformations} and calibrational freedom is hidden in the choise of basis. Using Schmidt decomposition\footnote{Schmidt decomposition can be performed in finite dimension, or if the Hamiltonian is compact, which is not automatic in quantum mechanics. What's more, the Hamiltonian is usually not even bounded. Anyway, for simple systems with bounded energy we can assume so.}, we can write the unitar transformation $\U$ between two systems $S$ and $\tilde{S}$
\begin{equation}
    \kpsi = \sum_{m,n}\psi_n \U_{nm}^*\ket{m(\llambda)} = \sum_m \overbrace{\tilde{\psi}_m(\llambda)}^{\braket{m(\llambda)|\psi}}\ket{m(\llambda)}.
\end{equation}
We can interpret this in \emph{active} resp. \emph{passive} way, i.e. as a transformation between two different states describing different systems, resp. as a transformation between different observers with different choice of basis. In quantum mechanics the more usual terms are \emph{Heisenberg} resp. \emph{Schrödinger} picture, but we will stick to the interpretation terminology, which makes the psysical meaning clearer.

In active interpretation we can assume the unitary transformation from some basis of $\HH(\llambda)$ to the basis comoving with the state\footnote{Comoving in a sence, that the wavefunction is not changing in this coordinate system.}, noted with \emph{tilde}
\begin{equation}
    \U(\llambda): \ket{\tilde\psi(\llambda)} \rightarrow \kpsi.
    \label{eq:transformationU}
\end{equation}
We can define adiabatic potentials analogically to the classical case as
\begin{equation}
    i\hbar\partial_\lambda \ket{\tilde{\psi}(\llambda)} = i\hbar \partial_\lambda\left(\U^+(\llambda)\ket{\psi} \right)= \underbrace{i\hbar\left(\partial_\lambda \U^+(\llambda)\right)\U(\llambda)}_{-\tilde{\AA_\lambda}}\ket{\tilde{\psi}(\llambda)},
\end{equation}
which can be rewritten to non-tilde basis as
\begin{equation}
    \begin{split}
        \AA_\lambda&=\U(\llambda)\tilde{\AA_\lambda}\U^+(\llambda) = -i\hbar\big(\U(\llambda)\partial_\lambda \U^+(\llambda)\big) =\\
        &= -i\hbar\big(\partial_\lambda(\underbrace{U^+(\llambda)U(\llambda)}_{\mathds{1}})-\partial_\lambda(U(\llambda))U^+(\llambda) \big) =i\hbar \big(\partial_\lambda U(\llambda))U^+(\llambda).
    \end{split}
\end{equation}
Thus we get equations for adiabatic potentials
\begin{align}
    \AA_\lambda&=i\hbar \big(\partial_\lambda U(\llambda))U^+(\llambda)
    \label{eq:adiabaticPotential}\\
    \tilde{\AA_\lambda} &= -i\hbar\left(\partial_\lambda \U^+(\llambda)\right)\U(\llambda)
    \label{eq:adiabaticPotentialTilde}
\end{align}
These potencials are Hermitean (omitting referrence to $\llambda$ in brackets)
\begin{equation}
     \tilde{\AA_\lambda}^+=i\hbar U^+\partial_\lambda\U=-i\hbar\partial_\lambda\U^+\U = \tilde{\AA_\lambda} ,
\end{equation}
analogically holds for $\AA_\lambda$ and using the eigenbasis of $\HH$, the matrix elements are
\begin{equation}
    \bra{n}\tilde{\AA_\lambda}\ket{m}=i\hbar\bra{n}\U^+\partial_\lambda\U\ket{m} = i\hbar\bra{\tilde n(\lambda)}\partial_\lambda\ket{\tilde m(\lambda)}.
\end{equation}
and because
\begin{equation}
    \bra{\tilde n(\lambda)}\AA_\lambda\ket{\tilde m(\lambda)}= \bra{n}\tilde{\AA_\lambda}\ket{m},
\end{equation}
we get
\begin{equation}
    \A_\lambda = i\hbar\partial_\lambda.
    \label{eq:adiabaticPotentialDefinition}
\end{equation}
It's good to point out, that we were applying tilde operators to non-tilde states et vice versa. This can be justified only if those states belong to the same Hilbert space, or in the geometrical language, to the same manifold, which we can define as $\HH_{full}=\HH\otimes\tilde\HH$.


\section{Adiabatic transformations}
In this chapter, we will be dealing with the system described by finite-dimensional Hamiltonian $\HH=\HH(\llambda)$ which drives the system according to \Schrodinger equation from some initial state $\ket{n(\llambda(0))}$ to $\ket{n(\llambda(t))}$. The action minimalization theorem states, that there exists some path minimiying action, which allows us to reduce the dimantion of $\M$ from the $\R^n$ to $1$ using parametrisation $\gamma(\lambda)$, so we get $\HH=\HH(\lambda)$.

\subsection{Adiabatic potential}
Important knoledge about symmetries of the system is encoded in canonical transformations, or in quantum mechanics more commonly reffered to as \emph{unitar transformations}. In our case, the generators of such canonical transformations are adiabatic potentials. In case of the Hamiltonian $\H(\llambda)$ and it's adiabatic transformation $\H(\llambda+\d \llambda)$, we get
\begin{equation}
    [\HH(\llambda),\HH(\llambda+\d \llambda)]=0,
\end{equation}
meaning Hamiltonian commutes with it's cannonically transformed version.\footnote{This can be easily reformulated to the world of classical physics, where the commutator is replaced by Poisson bracket.}


\subsection{Adiabatic transformation}
\citep{kolodrubez}[chap. 2.3]
As was mentioned in the introduction to this chapter, one way to change the system parameters without exciting it is to change the driving parameter slowly enough. The meaning of the word "slow" clears up next theorem.
\begin{thm}[Adiabatic theorem]
    \label{adiabaticTheorem}
    For Hamiltonian $\HH$ varying in the time range $T$, the solution of the Schrödinger equation 
    $$\HH(t)\ket{\psi_n(t)} = E_n(t)\ket{\psi_n(t)}$$
    with initial condition in x-representation $\braket{x|\psi(t=0}=\psi_n(x,0)$ can be approximated as
    \begin{equation}
      ||\psi(t) - \psi_{ad}(t)||\approx o\left(\frac{1}{T}\right)
    \end{equation}
    for \emph{adiabatic state}
    \begin{equation}
        \ket{\psi_{ad}}= e^{\theta_n(t)}e^{\gamma_n(t)}\ket{\psi(t)},
    \end{equation}
    where we define \emph{nongeometrical phase} induced by energy transitions,
    $$\theta_n(t)\equiv -\frac{1}{\hbar}\int_0^t E_n(\tau)\d \tau$$
    and \emph{geometrical phase}, also called \emph{Berry phase}
        $$\gamma_n(t)\equiv \int_0^t \underbrace{i\braket{\psi_n(\tau)|\partial_\lambda\psi_n(\tau)}}_{\nu_n(\tau)} \d \tau .$$
\end{thm}
\begin{myproof}
    TBD (na wiki je)
\end{myproof}
Assume differentiable and non-singular Hamiltonian $\HH(\llambda)$ with degenerate basis $\{\ket{m,\llambda}\}_m$ called the \emph{adiabatic basis}. This is generally the family of adiabatically connected eigenstates\footnote{In the case of energy level crossing, the eigenstates are not unified, because transition between them is not adiabatical.} The transition amplitude between states for adiabatic change is
\begin{equation}
    0=\bra{m}\HH\ket{n} \quad \text{pro }n\neq m.
\end{equation}
This can be driven along some curve $\gamma(\lambda)$, i.e. differentiated by $\partial_\lambda$:
\begin{equation}
    \begin{split}
        0&=\bra{\partial_\lambda m}\HH\ket{n}+ \bra{m}\partial_\lambda\HH\ket{n}+ \bra{m}\HH\ket{\partial_\lambda n}\\
        &=E_n\braket{\partial_\lambda m|n} + E_m\braket{m|\partial_\lambda n}+ \bra{m}\partial_\lambda \HH\ket{n}\\
        &= (E_m-E_n)\underbrace{\braket{m|\partial_\lambda n}}_{-\frac{i}{\hbar}\bra{m}\AA_\lambda\ket{n}} + \braket{m|\partial_\lambda\HH|n},
    \end{split}
\end{equation}
where $\HH$, $\ket{n}$, $\ket{m}$ and $E_n$ are functions of $\lambda$.

In matrix form, we can rewrite this equation as
\begin{equation}
    i\hbar\partial_\lambda\HH=[\AA_\lambda,\HH]-i\hbar \hat{M}_\lambda\qquad \text{for } \hat{M}_\lambda\equiv -\sum_n\pder{E_n(\lambda)}{\lambda}\ket{n(\lambda)}\bra{n(\lambda)}.
\end{equation}
$\hat{M}$ is diagonal in energetic basis and it's elements has meaning of \emph{generalized force}, which correspond to corresponding energetic states. We can easily see that $[\HH,\hat{M}]=0$, implying
\begin{equation}
    [\HH,i\hbar\partial_\lambda\HH-[\AA_\lambda,\HH]]=0.
    \label{eq:komutation}
\end{equation}
This can be used as the definition for \emph{counterdiabatic potential} $\AA_\lambda$. The strength of this equation lies in the fact, that it finds counterdiabatic potential without the need of Hamiltonian diagonalisation.

\section{Counterdiabatic driving}
\citep{kolodrubez}[page 15--17]
Again consider two bases consisting of eigenstates of Hamiltonian $\HH=\HH(\lambda(t))$. $B(t)$ for external observer and $\tilde B(t)$ for frame activelly transformed by Hamiltonian (moving frame), in which $\tilde \HH(t)$ is diagonal. Transforming vectors in Schrödinger equation
\begin{equation}
    i\hbar \der{}{t}\ket{\psi(t)} = \HH(\lambda(t))\ket{\psi(t)}
\end{equation}
to moving frame using unitary operator for time varying Hamiltonian (compare to eq. \ref{eq:transformationU}
\begin{equation}
    \U(\lambda(t)): \ket{\tilde\psi(\lambda(t))} \rightarrow \ket{\psi(t)}.
    \label{eq:transformationUtimeDependentH}
\end{equation}
and using \emph{dot} notation for time derivative, we get
\begin{align}
    i\hbar \der{}{t}(\U\ket{\tilde \psi}) &= \HH\U\ket{\tilde \psi} \\
    i\hbar \dot \lambda\partial_\lambda\U\ket{\tilde \psi} + i\hbar \U\der{}{t}\ket{\tilde \psi} &= \HH\U\ket{\tilde \psi}.
\end{align}
This can be rewritten using adiabatic potential from eq. \ref{eq:adiabaticPotentialDefinition}) as
\begin{equation}
    i\hbar \der{}{t}\ket{\tilde{\psi}} = \left[\U^+\HH\U-\dot{\lambda}\tilde{\AA_\lambda}\right]\ket{\tilde{\psi}} = \left[\tilde \HH-\dot{\lambda}\tilde{\AA_\lambda}\right]\ket{\tilde{\psi}} = \tilde{\HH}_m \ket{\tilde{\psi}}.
\end{equation}
Hamiltonian in moving frame is $\tilde\HH(t)=\U^+(\lambda(t))\HH(\lambda(t))\U(\lambda(t))$ and the term $-\dot{\lambda}\tilde{\AA_\lambda}$ is called \emph{Galilean}. To Hamiltonian in moving frame $\HH_m = \HH-\dot{\lambda}\AA_\lambda$ we can add \emph{counterdiabatic element} $\dot{\lambda}\AA_\lambda$ and the only remaining element is $\HH$, which does not excite the system.

\section{Approximations of adiabatic potentials}
Adiabatic potentials can be calculated from the principal of minimal action, which leads to variational method.

If the difference between eigenstates of $\HH$ is small, or generalized force between some states is zero, the computation of the adiabatic potential is numerically unstable. The knoledge of exact adiabatic potential would allow to maintain the system in the ground state thus not exciting it, as th Eigenstate thermalization hypotheses states.

\begin{hypot}[Eigenstate thermalization hypotheses]
  For the difference between eigenstates of $\HH$ and extensive thermodynamic entropy $S$, it holds that
    \begin{equation}
    E_n-E_m\propto \exp\left(\frac{S}{2}\right).
  \end{equation}
  If the states are close, better approximation would be $E_n-E_m\propto \exp(S)$. For matrix elements it holds, that they vanish exponentially with the characteristic scale of the system $a$, i.e.
  \begin{equation}
    \bra{m}\AA_\lambda\ket{n} = i\hbar\frac{\braket{m|\partial_\lambda \HH|n}}{E_m-E_n} \propto \exp(-a).
    \label{eq:thermalizationMatrixElements}
\end{equation}
\end{hypot}
Fortunatelly in the limit "number of particles" $\rightarrow \infty$ the expression in eq. \ref{eq:thermalizationMatrixElements} converges.

\subsection{Variational methods}
