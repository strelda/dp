\chapter{Pure mathematical introduction}
The modern approach to the closed system dynamics is using differential geometry formalism. Before we get to the quantum mechanics itself, let's breathly define the formalism recapitulate some definitions of this part of mathematics.

Let's have a manifold $\M$ and curves 
$$\gamma:\R \overset{open}{\supset} I \rightarrow \M \qquad \xi\mapsto \gamma(\xi).$$ 
The space of functions is $\F(\M)\equiv\{f:\M\rightarrow \R\}$, where 
$$f:\M\rightarrow U\overset{open}{\subset} \R \qquad x\mapsto f(x).$$
To define \emph{vectors} on $\M$, we need to make sence of the \emph{direction}. It is defined using curves satisfying 
$$\gamma_1(0)=\gamma_2(0)\equiv P$$
$$\der{}{t}x^i(\gamma_1(t))\big|_{t=0}=\der{}{t}x^i(\gamma_2(t))\big|_{t=0}.$$
Taking the equivalence class created by those two rules, sometimes noted as $[\gamma]=v$, we have element of the tangent space to $\M$. We will use standart notation for the tangent space to $\M$ in some point $xP$ as $\T_P\M$ and contangent space as $\T^*_P\M$. Unifying all those spaces over all $x$ we get tangent and cotangent bundle, $\TT\M$ and $\TT^*\M$ respective. To generalize this notation to higher tensors, we denote $\T_P\M\in \TT^1\M$, $\T^*_P\M\in \TT_1\M$, thus the space of $p-$times contravariant and $q-$times covariant tensors is denoted $\TT^p_q\M$.

Using the congruence of the curves on $\M$, the expression 
\begin{equation}
    \der{}{\xi}f\circ \gamma(\xi)\Big|_{\xi=0}
\end{equation}
has a good meaning and we can define the \emph{derivative} in some $P\in\M$ as
\begin{equation}
    \bm v: \F(\M)\rightarrow \R \qquad f\mapsto \bm v[f]\equiv \frac{\d f(\gamma(\xi))}{\d \xi}\Big|_P \equiv \partial_\xi\Big|_P f .
\end{equation}
It holds, that $\bm v\in \T_P\M$ and can be expressed as the \emph{derivative in direction},\footnote{
        The direction itself is usually denoted as
        \begin{equation}
            \frac{\D}{\d\bm\alpha}\gamma(\xi),
        \end{equation}
        where the big D notation is used to point out that it's not a classical derivative, but it maps curves to some entirely new space of directions.
    } 
which can be understood in coordinates as
\begin{equation}
    \bm v[f] = \der{}{\bm v} f\circ \gamma(\xi)\Big|_{\xi=0}=v^\mu\der{}{x^\mu} f(\bm x)\Big|_{P}.
\end{equation}




To get some physical application, we need to define one strong structure on manifolds -- differentiable metric tensor $g_{\mu\nu}\in\TT^0_2\M$ -- so the covariant derivatives and parallel transport are well defined everywhere. 

\section{Covariant derivative and parallel transport}
Affine connection can be expressed as
\begin{equation}
    \Gamma^{\alpha}_{\mu\nu} = \frac{1}{2}g^{\alpha \beta}\left(g_{\beta\mu,\nu}+g_{\nu\beta,\mu}-g_{\mu\nu,\beta}\right),
\end{equation}
where we used comma notation for the coordinate derivative.
The covariant derivative of $\bm a\in \T_P\M$ is then defined
\begin{equation}
    \Der{a^\mu}{x^\nu}=a_{\;,\nu}^\mu-\Gamma^\mu_{\alpha\beta} x^\alpha a^\beta 
\end{equation}
and for $\bm\alpha\in \T_P^*\M$ it is
\begin{equation}
    \Der{\alpha_\mu}{x^\nu}=\alpha_{\mu,\nu}-\Gamma^\alpha_{\mu\beta}x^\beta \alpha_\alpha 
\end{equation}

The vector $v\in \T_P\M$ is said to be parallel transported along curve $\gamma(\lambda)$, if it's covariant derivative
\begin{equation}
    \Der{v^\mu}{\xi}=0
\end{equation}
vanishes along $\gamma$.

\section{Antisymmetric tensors and wedge product}
p-form $A\in \TT_p \M$ is called \emph{antisymmetric}, if changing the order of the indices has impact only on the sign, symbolically
$$A_{i_1\dots i_p} = \sign(\sigma)A_{i_{\sigma_1}\dots i_{\sigma_p}},$$
where $\sigma$ is some permutation.\emph{Antisymmetrisation} is defined as a normalized sum over all permutation
\begin{equation}
    A^{[i_1\dots i_p]}\equiv \frac{1}{p!}\sum_\sigma A^{[i_{\sigma_1}\dots i_{\sigma_p}]}. 
\end{equation}

The \emph{wedge product} of $A\in \TT_p \M$ and $B\in \TT_q \M$ is antisymmetrisation of the tensor product in the sence
\begin{equation}
    A\wedge B\equiv \frac{(p+q)!}{p!q!} A^{[i_1\dots i_p}\otimes B^{i_1\dots i_q]}
\end{equation}



\chapter{Physical introduction}
Now we will assign some physical background to the structure defined in the first chapter.

Assume manifold $\M$ generated by eigenstates of some closed system Hamiltonian $\HH(\llambda)$, meaning the Hamiltonian is bounded and dimension of the space is finite. Let's assume the existence of $\mathcal{C}^1$ mapping\footnote{is continuous to the first derivative} (parametrisation) $B: \M\rightarrow \bm\llambda\equiv (\lambda^1,\dots,\lambda^n)\in\R^n$. Therefore we will denote eigenvectors $\ket{\iota(\bm\llambda)}$ and their energies $E(\bm\llambda)$.

Now we need to find some reasonable way to measure the distance on $\M$. Our first guess might be
\begin{equation}
    \d \tilde{s}^2 = \braket{\iota(\bm\llambda+\d\bm\llambda)|\iota(\bm\llambda+\d\bm\llambda)} = 1-2\Re{\braket{\iota(\bm\llambda+\bm\d\llambda)|\iota(\bm\llambda)}}.
\end{equation}
This is not \emph{gauge dependent}, meaning that it depends on our choice of the wave phase. Gauge independent choise would be for example 
\begin{equation}
    f=\braket{\iota(\bm\llambda+\d\bm\llambda)|\iota(\bm\llambda)},
\end{equation}
sometimes refered to as the \emph{fidelity}. We can see it's physical meaning imagining \emph{quantum quench} (rapid change of some Hamiltonian parameters), in which case $f^2$ is the probability that system will remain in the new ground state. $1-f^2$ is therefore probability to excite the system during this quench, which leads to the definition of \emph{distance on $\M$}
\begin{equation}
    \d s \equiv 1-f^2= 1-\left|\braket{\iota(\bm\llambda+\d\bm\llambda)|\iota(\bm\llambda)}\right|.
\end{equation}
Using $\d s^2 = g_{\mu\nu}\d \llambda^\mu \d\llambda^\nu+\O(\llambda^3)$, we get the metric tensor
\begin{equation}
    g_{\mu\nu}^{(i)}(\bm\llambda) = \Re\left(\braket{\partial_{\lambda^\mu}\iota(\bm\llambda)|\partial_{\lambda^\nu}\iota(\bm\llambda)} - \braket{\partial_{\lambda^\mu}\iota(\bm\llambda)|\iota(\bm\llambda)}\braket{\iota(\bm\llambda)|\partial_{\lambda^\nu}\iota(\bm\llambda)}\right).
    \label{eq:geom.tensorDefinition}
\end{equation}

Let's have a initial state described by Hamiltonian $\H_\iota=\H(\llambda)$ in eigenstate $\ket{\iota(\llambda)}$, which undergoes the change of parameters $\llambda\rightarrow \llambda+\d \llambda$ resulting in the Hamiltonian $\H_f$ with eigenstates $\ket{\psi_n(\llambda+\d \llambda)}$, $n\in \{1,\dots,dim(\H_f)\}$. Probability amplitude of going to some specific excited state is
\begin{equation}
    \begin{split}
        a_n&=\braket{\psi_n(\llambda+\d\llambda)|\iota(\llambda)}\approx \d\lambda^\mu\braket{\partial_\mu \psi_n(\llambda)|\iota(\llambda} \\
        &= -\d\lambda^\mu\braket{\psi_n(\llambda)|\partial_\mu|\iota(\llambda)}\equiv -\d\lambda^\mu\braket{n|\partial_\mu|\iota},
    \end{split}
\end{equation}
where we introduced shortend notation for eigenstates of the Hamiltonian $\H_0$. If we introduce the \emph{gauge potential}, a.k.a \emph{calibration potential} as
\begin{equation}
    \AA_\mu\equiv i\hbar \partial_{\mu}
\end{equation}
and rescale units to $\hbar=1$, as we will use further on, we get
\begin{equation}
   a_n=\sum_\mu i\braket{n|\AA_\mu |\iota}\d\lambda^\mu,
\end{equation}
which has meaning of matrix elements of the gauge potential. Probability of the excitation i.e. transition to any state $n>0$ is then
\begin{equation}
    \begin{split}
        \sum_{n\neq 0}|a_n|^2&=  \sum_{n\neq 0} \d \lambda^\mu \d \lambda^\nu\braket{\iota|\AA_\mu|n}\braket{n|\AA_\nu|\iota}+\O(|\d \lambda^3|) = \d \lambda^\mu \d \lambda^\nu\braket{\iota|\AA_\mu \AA_\nu|\iota}_c\\
        &= \d \lambda^\mu \d \lambda^\nu\chi_{\mu\nu}+\O(|\d \lambda^3|)=\d s^2+\O(|\d \lambda^3|),
    \end{split}
\end{equation}
where we defined \emph{connected correlation function}, or \emph{covariance}
\begin{equation}
    \braket{\iota|\AA_\mu\AA_\nu|\iota}_c\equiv \braket{\iota|\AA_\mu\AA_\nu|\iota} - \braket{\iota|\AA_\mu|\iota}\braket{\iota|\AA_\nu|\iota}.
    \label{eq:covariance}
\end{equation}
If we leave out $\hbar$, we have the \emph{geometric tensor}\footnote{sometimes defined directly as the expression in eq. \ref{eq:covariance}}
\begin{equation}
    \chi_{\mu\nu}\equiv \braket{\partial_\mu \iota|\partial_\nu \iota}_c = \braket{\partial_\mu \iota|\partial_\nu \iota} - \braket{\partial_\mu \iota|\iota}\braket{\iota|\partial_\nu \iota},
\end{equation}
where $\ket{\partial_\nu \iota}\equiv\partial_\nu \ket{ \iota}$. Because $\chi$ is Hermitian ($\chi_{\mu\nu}=\chi^*_{\nu\mu}$), only the symmetric part adds up to the distance between states 
\begin{equation}
    \d s^2= g_{\mu\nu}\d \lambda^\mu \lambda^\nu= \chi_{\mu\nu}\d \lambda^\mu \lambda^\nu.
\end{equation}
 and only the symmetric part determines the distance between the states. Therefore it's practical to decompose it as
\begin{equation}
    \chi_{\mu\nu} \equiv g_{\mu\nu} - \textcolor{red}{i\frac{1}{2}} \nu_{\mu\nu},
\end{equation}
where the \emph{Fubini-Study tensor}, as it's called, is
\begin{equation}
    g_{\mu\nu} = \frac{\chi_{\mu\nu}+\chi_{\nu\mu}}{2} = \Re\braket{\partial_\mu i|\partial_\nu i}_c = \textcolor{red}{\Re \sum_{i\neq j}\frac{\braket{\iota|\pder{\H}{\lambda^\mu}|j}\braket{j|\pder{\H}{\lambda^\nu}|\iota}}{(E_i-E_j)^2}},
    \label{eq:geom.tensorREdefinition}
\end{equation}
and the \emph{curvature tensor} a.k.a. \emph{Berry curvature} is
\begin{equation}
    \begin{split}
        \nu_{\mu\nu} = 2 i(\chi_{\mu\nu}-\chi_{\nu\mu})&= \Im\braket{\iota|[\overleftarrow{\partial}_\nu,\partial_\mu]|\iota}_c \\
        &= -2 \Im \sum_{i\neq j}\frac{\braket{\iota|\pder{\H}{lambda^\mu}|j}\braket{j|\pder{\H}{lambda^\nu}|\iota}}{(E_i-E_j)^2},
    \end{split}
\end{equation}
where $\overleftarrow{\partial}_\nu$ is the derivative of the covector on the left.

Fubini-Study tensor can be seen as the Pull-back of the elements of the full Hilbert space to $\M$. 


Next we define \emph{Berry connection}
\begin{equation}
    A_\mu\equiv \braket{\iota|\AA_\mu|\iota},
\end{equation}
which empovers us to write
\begin{equation}
    \nu_{\mu\nu} = \partial_\mu A_\nu-\partial_\nu A_\mu
\end{equation}
and \emph{Berry phase}
    \footnote{
        The reasonability of this definition can be seen, if we assume the ground state of a free particle
            $\braket{\bm{x}|\iota}=\iota(\bm{x},\llambda)= |\iota(\bm{x})|e^{i\phi(\llambda)}$,
        then the Berry connection is
        \begin{equation}
            A_\mu=-\int \d \bm{x}|\iota|^2\partial_\mu \phi = -\partial_\mu \phi
        \end{equation} 
        and Berry phase
        \begin{equation}
            \phi_B=\oint_\mathcal{C} \partial_\mu \phi \d \lambda^\mu,
        \end{equation}
        which represents total phase accumulated by the wavefunction. It is really the analogy for Berry phase in classical mechanics, which for example for the Faucolt pendulum on one trip around the Sun makes $\phi_B=2\pi$
    }
\begin{equation}
    \phi_B\equiv-\oint_\mathcal{C} A_\mu \d \lambda^\mu=\int_\mathcal{S} F_{\mu\nu}\d \lambda^\mu \wedge \d\lambda^nu,
\end{equation}
where we used the Stokes theorem defining, that the curve $\mathcal{C}$ surrounds some area $\mathcal{S}$.

Wave-functions are elements of the tangent bundle $\TT\in \M$, the gauge potentials are affine connections defining the parallel transport. Covariant derivative is
\begin{equation}
    D_\mu=\partial_\mu+\frac{i}{\hbar}\AA_\mu,
\end{equation}
which yields $D_\mu\ket{\psi_n}=0$ for every eigenstate, which encloses the circle and \textcolor{red}{justifies our initial choise for the distance on $\M$.}