%%% The main file. It contains definitions of basic parameters and includes all other parts.

%% Settings for single-side (simplex) printing
% Margins: left 40mm, right 25mm, top and bottom 25mm
% (but beware, LaTeX adds 1in implicitly)
\documentclass[12pt,a4paper]{report}
\setlength\textwidth{145mm}
\setlength\textheight{247mm}
\setlength\oddsidemargin{15mm}
\setlength\evensidemargin{15mm}
\setlength\topmargin{0mm}
\setlength\headsep{0mm}
\setlength\headheight{0mm}
% \openright makes the following text appear on a right-hand page
\let\openright=\clearpage

%% Settings for two-sided (duplex) printing
% \documentclass[12pt,a4paper,twoside,openright]{report}
% \setlength\textwidth{145mm}
% \setlength\textheight{247mm}
% \setlength\oddsidemargin{14.2mm}
% \setlength\evensidemargin{0mm}
% \setlength\topmargin{0mm}
% \setlength\headsep{0mm}
% \setlength\headheight{0mm}
% \let\openright=\cleardoublepage

%% Generate PDF/A-2u
\usepackage[a-2u]{pdfx}

%% Character encoding: usually latin2, cp1250 or utf8:
\usepackage[utf8]{inputenc}

%% Prefer Latin Modern fonts
\usepackage{lmodern}

%% Further useful packages (included in most LaTeX distributions)
\usepackage{amsmath}        % extensions for typesetting of math
\usepackage{amsfonts}       % math fonts
\usepackage{amsthm}         % theorems, definitions, etc.
\usepackage{bbding}         % various symbols (squares, asterisks, scissors, ...)
\usepackage{bm}             % boldface symbols (\bm)
\usepackage{graphicx}       % embedding of pictures
\usepackage{fancyvrb}       % improved verbatim environment
\usepackage{natbib}         % citation style AUTHOR (YEAR), or AUTHOR [NUMBER]
\usepackage[nottoc]{tocbibind} % makes sure that bibliography and the lists
			    % of figures/tables are included in the table
			    % of contents
\usepackage{dcolumn}        % improved alignment of table columns
\usepackage{booktabs}       % improved horizontal lines in tables
\usepackage{paralist}       % improved enumerate and itemize
\usepackage{xcolor}         % typesetting in color

\usepackage{dsfont}
\usepackage{braket}
\usepackage{tikz} % Export grafů z RStudia
\usepackage{float} %použití [H] u figur, umístění na přesné místo
\usepackage{mathtools} %pro \dcases
\usepackage{subcaption}


%\usepackage{array}
%\usepackage{dsfont}

% \usepackage{amsmath}        % rozšíření pro sazbu matematiky
% \usepackage{amssymb}        %přidáno, pokud se něco kazí, tak to může být tímto----------------------------------------------
% \usepackage{amsfonts}       % matematické fonty
% \usepackage{amsthm}         % sazba vět, definic apod.
% \usepackage{bbding}         % balíček s nejrůznějšími symboly
% 			    % (čtverečky, hvězdičky, tužtičky, nůžtičky, ...)
% \usepackage{bm}             % tučné symboly (příkaz \bm)
% \usepackage{graphicx}       % vkládání obrázků
% \usepackage{fancyvrb}       % vylepšené prostředí pro strojové písmo
% \usepackage{indentfirst}    % zavede odsazení 1. odstavce kapitoly
% \usepackage{natbib}         % zajištuje možnost odkazovat na literaturu
% 			    % stylem AUTOR (ROK), resp. AUTOR [ČÍSLO]
% \usepackage[nottoc]{tocbibind} % zajistí přidání seznamu literatury,
%                             % obrázků a tabulek do obsahu
% \usepackage{icomma}         % inteligetní čárka v matematickém módu
% \usepackage{dcolumn}        % lepší zarovnání sloupců v tabulkách
% \usepackage{booktabs}       % lepší vodorovné linky v tabulkách
% \usepackage{paralist}       % lepší enumerate a itemize
% \usepackage{xcolor}         % barevná sazba


%%% Basic information on the thesis

% Thesis title in English (exactly as in the formal assignment)
\def\ThesisTitle{Some funny small things looking for counterdiabatic elements
}

% Author of the thesis
\def\ThesisAuthor{Jan Střeleček}

% Year when the thesis is submitted
\def\YearSubmitted{2022}

% Name of the department or institute, where the work was officially assigned
% (according to the Organizational Structure of MFF UK in English,
% or a full name of a department outside MFF)
\def\Department{Institute of Particle & Nuclear Physics}

% Is it a department (katedra), or an institute (ústav)?
\def\DeptType{Institute}

% Thesis supervisor: name, surname and titles
\def\Supervisor{prof. RNDr. Pavel Cejnar, Dr., DSc.}

% Supervisor's department (again according to Organizational structure of MFF)
\def\SupervisorsDepartment{Institute of Particle & Nuclear Physics}

% Study programme and specialization
\def\StudyProgramme{Theoretical physics}
\def\StudyBranch{hmmmmm}

% An optional dedication: you can thank whomever you wish (your supervisor,
% consultant, a person who lent the software, etc.)
\def\Dedication{%
Dedication.
}

% Abstract (recommended length around 80-200 words; this is not a copy of your thesis assignment!)
\def\Abstract{%
Abstract.
}

% 3 to 5 keywords (recommended), each enclosed in curly braces
\def\Keywords{%
{key} {words}
}

%% The hyperref package for clickable links in PDF and also for storing
%% metadata to PDF (including the table of contents).
%% Most settings are pre-set by the pdfx package.
\hypersetup{unicode}
\hypersetup{breaklinks=true}

% Definitions of macros (see description inside)
\DeclareMathOperator{\Tr}{\textrm{Tr}}
\DeclareMathOperator\End{End}

\renewcommand{\d}{\ensuremath{\mathrm{d}}}
\newcommand{\D}{\ensuremath{\mathrm{D}}}
\newcommand{\pder}[2]{\frac{\partial #1}{\partial #2}}
\newcommand{\der}[2]{\frac{\mathrm{d} #1}{\mathrm{d} #2}}
\newcommand{\Der}[2]{\frac{\mathrm{D} #1}{\mathrm{d} #2}}
\newtheorem{thm}{Theorem}

\newcommand{\M}{\mathcal{M}}
\renewcommand{\P}{\mathcal{P}}
\newcommand{\R}{\mathbb{R}}
\newcommand{\C}{\mathbb{C}}
\newcommand{\N}{\mathbb{N}}
\newcommand{\F}{\mathcal{F}}
\newcommand{\TT}{\mathcal{T}}
\renewcommand{\O}{\mathcal{O}}

\newcommand{\Id}{\mathbbm{1}}
\newcommand{\A}{\mathcal{A}}
\newcommand{\llambda}{{\bm\lambda}}
\renewcommand{\AA}{\mathcal{\widehat{A}}}
\newcommand{\U}{\hat{U}}
\renewcommand{\H}{\mathcal{H}}
\newcommand{\HH}{\hat{H}}
\newcommand{\J}{\hat{J}}
\newcommand{\kpsi}{\ket{\psi}}
\newcommand{\kphi}{\ket{\phi}}
\newcommand{\kpsit}{\ket{\psi(t)}}
\newcommand{\kpsilt}{\ket{\psi(\llambda(t))}}
\newcommand{\up}{\ket{\uparrow}}
\newcommand{\dn}{\ket{\downarrow}}
\newcommand{\ch}{\hat{\chi}}
\newcommand{\Schrodinger}{Schrödinger }
\newcommand{\PH}{\mathcal{PH}}
\newcommand{\Z}{\mathbb{Z}}
\newcommand{\Span}{\text{Span}}
\renewcommand{\Re}{\text{Re}}
\newcommand{\FM}{\mathcal{FM}}

\newcommand{\expsm}{e^{-\frac{i \omega}{2}\hat\sigma_y  t}}
\newcommand{\expsp}{e^{\frac{i \omega}{2}\hat\sigma_y  t}}
\newcommand{\UU}{\hat U}


\DeclareMathOperator{\spec}{\sigma}





\usepackage{xcolor}
\definecolor{red}{rgb}{0.9,0.05,0.05}
\definecolor{redd}{rgb}{0.7,0.1,0.1}
\definecolor{reddd}{rgb}{0.7,0.2,0.0}

\definecolor{green}{rgb}{0.05,0.9,0.05}
\definecolor{greenn}{rgb}{0.2,0.65,0.2}
\definecolor{greennn}{rgb}{0.2,0.8,0.7}

\definecolor{blue}{rgb}{0.05,0.05,0.9}
\definecolor{bluee}{rgb}{0.2,0.2,0.6}
\definecolor{blueee}{rgb}{0.55,0.55,0.9}


\newcommand{\red}[1]{\textcolor{red}{#1}}
\newcommand{\redd}[1]{\textcolor{redd}{#1}}
\newcommand{\reddd}[1]{\textcolor{reddd}{#1}}

\newcommand{\green}[1]{\textcolor{green}{#1}}
\newcommand{\greenn}[1]{\textcolor{greenn}{#1}}
\newcommand{\greennn}[1]{\textcolor{greennn}{#1}}

\newcommand{\blue}[1]{\textcolor{blue}{#1}}
\newcommand{\bluee}[1]{\textcolor{bluee}{#1}}
\newcommand{\blueee}[1]{\textcolor{blueee}{#1}}

\newcommand{\gray}[1]{\textcolor{gray}{#1}}



\newcommand{\leftsquigarrow}{\reflectbox{$\rightsquigarrow$}}
\newcommand{\curlyrightarrow}[1]{\overset{\rightsquigarrow}{#1}}
\newcommand{\curlyleftarrow}[1]{\overset{{\leftsquigarrow}}{#1}}

% Title page and various mandatory informational pages
\begin{document}
\include{title}

%%% A page with automatically generated table of contents of the master thesis

\tableofcontents

%%% Each chapter is kept in a separate file

\chapwithtoc{Some notes to the notation}

\begin{tabular} {@{}C{1.9cm}@{}p{8cm}@{}C{3.77cm}}
	\toprule
	\textbf{Symbol}& \textbf{Meaning}& \textbf{Defining formula}\\\bottomrule
	$\A$ & Gauge (calibrational) potential & $\A_\mu=i\hbar \partial_\mu$ \\
	$\mathbb{N}$ & Natural numbers, without zero \\
\bottomrule
\multicolumn{3}{l}{\footnotesize}
\end{tabular}

Mathematical spaces will be denoted \emph{mathcal} and operators with \emph{hat}.
\chapter*{\red{Introduction}}
\addcontentsline{toc}{chapter}{Introduction}
One of the unsolved problems of the quantum physics are quantum computers. There are many mathematical problems, which are solvable in exponential time on computers with classical bits, but are solvable in polynomial time on quantum computers. Essentially you prepare some initial state of qubits (these might be electron spins, or more recently, Josephson junctions are being used) and perform certain operations on them using \emph{quantum gates}. In the end you measure the qubits, causing the collapse of wave-function, and read the result. The first main problems in this area, is holding the superposition of qubits until all operations are performed. The second great problem is the quantum noise, either spontaneous emission of excited states, or interaction with the thermal basis of the surrounding. The impact of these effects can be seen on symmetrical experiment, in which we start with some state, let's say spin up. Perform any number of operation on it and then perform their inverse, leading to the same state, spin up. In perfect quantum computer, we would get the initial state with 100 \% accuracy. The problem is that due to noise we sometimes measure different state, in this case it would be spin down. The \emph{percentage of getting the result we want} is called the \emph{fidelity}.

This problem is of course more general. What is happening in the example above mathematically, is that we have interaction Hamiltonian between the qubits, thermal basis and quantum gates. The interaction with gates can be described by some Hamiltonian element with free parameter. Changing this parameter influences the qubit and \emph{drives} it to some final state, which will be measured. The theory of quantum driving, as created by physicists in the second half of 20. century, uses mathematical formalism which sometimes lacks on precise definitions. It can be formalized in a language of differential geometry, which basics are reminded in Chapter \ref{chap:mathIntro}. The theory of quantum driving itself is described in Chapter \ref{chap:driving}.

The important question in quantum computing is: "How to achieve the greatest \emph{final fidelity}, meaning \emph{how to prepare the state we want to prepare with the highest percentage}?" During the driving one might add some energy to the qubit, which leads to its excitation and possibly destroying the superposition. This can be avoided by many methods. There are a few methods how to avoid this excitation, described in Chapter \ref{chap:typesOfDriving}. The surprising fact is that not every sequence of quantum gates leads to the same fidelity. For example if one starts with \emph{spin up}, applying the $X$ or $Y$ gate has the same effect. Both result in \emph{spin down}, because these gates just rotate the spin in a Bloch sphere around corresponding axis ($x$, resp. $y$).

For some special drivings, such as driving using small \emph{quenches} (quick, but small change in driving parameter), one might get interested in \emph{ground state manifold properties}.

To understand the general fidelity driving, a simple two level system was analyzed in Chapter \ref{chap:twoLevelSystem}. Some driving phenomena are noticed here, which are demonstrated on the two analytically solvable drivings. Because with the Hamiltonian complexity, the driving complicates noticeably, it is important to understand the geometry of ground state manifolds first. The ground state manifold consists of all ground states of Hamiltonian with different driving parameter value. Until now not many implications of this structure is known. Some of them were theoreticized in previous works, some of them were developed here. In Chapter \ref{chap:groundStateManifoldDriving}, there is a general introduction of driving on ground state manifold. 



\red{another motivation is in biochemistry...}
\chapter{Motivation}
It's fun!
\chapter{Pure mathematical introduction}
The modern approach to the closed system dynamics is using differential geometry formalism. Before we get to the quantum mechanics itself, let's breathly define the formalism recapitulate some definitions of this part of mathematics.

Let's have a manifold $\M$ and curves 
$$\gamma:\R \overset{open}{\supset} I \rightarrow \M \qquad \xi\mapsto \gamma(\xi).$$ 
The space of functions is $\F(\M)\equiv\{f:\M\rightarrow \R\}$, where 
$$f:\M\rightarrow U\overset{open}{\subset} \R \qquad x\mapsto f(x).$$
To define \emph{vectors} on $\M$, we need to make sence of the \emph{direction}. It is defined using curves satisfying 
$$\gamma_1(0)=\gamma_2(0)\equiv P$$
$$\der{}{t}x^i(\gamma_1(t))\big|_{t=0}=\der{}{t}x^i(\gamma_2(t))\big|_{t=0}.$$
Taking the equivalence class created by those two rules, sometimes noted as $[\gamma]=v$, we have element of the tangent space to $\M$. We will use standart notation for the tangent space to $\M$ in some point $xP$ as $\T_P\M$ and contangent space as $\T^*_P\M$. Unifying all those spaces over all $x$ we get tangent and cotangent bundle, $\TT\M$ and $\TT^*\M$ respective. To generalize this notation to higher tensors, we denote $\T_P\M\in \TT^1\M$, $\T^*_P\M\in \TT_1\M$, thus the space of $p-$times contravariant and $q-$times covariant tensors is denoted $\TT^p_q\M$.

Using the congruence of the curves on $\M$, the expression 
\begin{equation}
    \der{}{\xi}f\circ \gamma(\xi)\Big|_{\xi=0}
\end{equation}
has a good meaning and we can define the \emph{derivative} in some $P\in\M$ as
\begin{equation}
    \bm v: \F(\M)\rightarrow \R \qquad f\mapsto \bm v[f]\equiv \frac{\d f(\gamma(\xi))}{\d \xi}\Big|_P \equiv \partial_\xi\Big|_P f .
\end{equation}
It holds, that $\bm v\in \T_P\M$ and can be expressed as the \emph{derivative in direction},\footnote{
        The direction itself is usually denoted as
        \begin{equation}
            \frac{\D}{\d\bm\alpha}\gamma(\xi),
        \end{equation}
        where the big D notation is used to point out that it's not a classical derivative, but it maps curves to some entirely new space of directions.
    } 
which can be understood in coordinates as
\begin{equation}
    \bm v[f] = \der{}{\bm v} f\circ \gamma(\xi)\Big|_{\xi=0}=v^\mu\der{}{x^\mu} f(\bm x)\Big|_{P}.
\end{equation}




To get some physical application, we need to define one strong structure on manifolds -- differentiable metric tensor $g_{\mu\nu}\in\TT^0_2\M$ -- so the covariant derivatives and parallel transport are well defined everywhere. 

\section{Covariant derivative and parallel transport}
Affine connection can be expressed as
\begin{equation}
    \Gamma^{\alpha}_{\mu\nu} = \frac{1}{2}g^{\alpha \beta}\left(g_{\beta\mu,\nu}+g_{\nu\beta,\mu}-g_{\mu\nu,\beta}\right),
\end{equation}
where we used comma notation for the coordinate derivative.
The covariant derivative of $\bm a\in \T_P\M$ is then defined
\begin{equation}
    \Der{a^\mu}{x^\nu}=a_{\;,\nu}^\mu-\Gamma^\mu_{\alpha\beta} x^\alpha a^\beta 
\end{equation}
and for $\bm\alpha\in \T_P^*\M$ it is
\begin{equation}
    \Der{\alpha_\mu}{x^\nu}=\alpha_{\mu,\nu}-\Gamma^\alpha_{\mu\beta}x^\beta \alpha_\alpha 
\end{equation}

The vector $v\in \T_P\M$ is said to be parallel transported along curve $\gamma(\lambda)$, if it's covariant derivative
\begin{equation}
    \Der{v^\mu}{\xi}=0
\end{equation}
vanishes along $\gamma$.

\section{Antisymmetric tensors and wedge product}
p-form $A\in \TT_p \M$ is called \emph{antisymmetric}, if changing the order of the indices has impact only on the sign, symbolically
$$A_{i_1\dots i_p} = \sign(\sigma)A_{i_{\sigma_1}\dots i_{\sigma_p}},$$
where $\sigma$ is some permutation.\emph{Antisymmetrisation} is defined as a normalized sum over all permutation
\begin{equation}
    A^{[i_1\dots i_p]}\equiv \frac{1}{p!}\sum_\sigma A^{[i_{\sigma_1}\dots i_{\sigma_p}]}. 
\end{equation}

The \emph{wedge product} of $A\in \TT_p \M$ and $B\in \TT_q \M$ is antisymmetrisation of the tensor product in the sence
\begin{equation}
    A\wedge B\equiv \frac{(p+q)!}{p!q!} A^{[i_1\dots i_p}\otimes B^{i_1\dots i_q]}
\end{equation}



\chapter{Physical introduction}
Now we will assign some physical background to the structure defined in the first chapter.

Assume manifold $\M$ generated by eigenstates of some closed system Hamiltonian $\HH(\llambda)$, meaning the Hamiltonian is bounded and dimension of the space is finite. Let's assume the existence of $\mathcal{C}^1$ mapping\footnote{is continuous to the first derivative} (parametrisation) $B: \M\rightarrow \bm\llambda\equiv (\lambda^1,\dots,\lambda^n)\in\R^n$. Therefore we will denote eigenvectors $\ket{\iota(\bm\llambda)}$ and their energies $E(\bm\llambda)$.

Now we need to find some reasonable way to measure the distance on $\M$. Our first guess might be
\begin{equation}
    \d \tilde{s}^2 = \braket{\iota(\bm\llambda+\d\bm\llambda)|\iota(\bm\llambda+\d\bm\llambda)} = 1-2\Re{\braket{\iota(\bm\llambda+\bm\d\llambda)|\iota(\bm\llambda)}}.
\end{equation}
This is not \emph{gauge dependent}, meaning that it depends on our choice of the wave phase. Gauge independent choise would be for example 
\begin{equation}
    f=\braket{\iota(\bm\llambda+\d\bm\llambda)|\iota(\bm\llambda)},
\end{equation}
sometimes refered to as the \emph{fidelity}. We can see it's physical meaning imagining \emph{quantum quench} (rapid change of some Hamiltonian parameters), in which case $f^2$ is the probability that system will remain in the new ground state. $1-f^2$ is therefore probability to excite the system during this quench, which leads to the definition of \emph{distance on $\M$}
\begin{equation}
    \d s \equiv 1-f^2= 1-\left|\braket{\iota(\bm\llambda+\d\bm\llambda)|\iota(\bm\llambda)}\right|.
\end{equation}
Using $\d s^2 = g_{\mu\nu}\d \llambda^\mu \d\llambda^\nu+\O(\llambda^3)$, we get the metric tensor
\begin{equation}
    g_{\mu\nu}^{(i)}(\bm\llambda) = \Re\left(\braket{\partial_{\lambda^\mu}\iota(\bm\llambda)|\partial_{\lambda^\nu}\iota(\bm\llambda)} - \braket{\partial_{\lambda^\mu}\iota(\bm\llambda)|\iota(\bm\llambda)}\braket{\iota(\bm\llambda)|\partial_{\lambda^\nu}\iota(\bm\llambda)}\right).
    \label{eq:geom.tensorDefinition}
\end{equation}

Let's have a initial state described by Hamiltonian $\H_\iota=\H(\llambda)$ in eigenstate $\ket{\iota(\llambda)}$, which undergoes the change of parameters $\llambda\rightarrow \llambda+\d \llambda$ resulting in the Hamiltonian $\H_f$ with eigenstates $\ket{\psi_n(\llambda+\d \llambda)}$, $n\in \{1,\dots,dim(\H_f)\}$. Probability amplitude of going to some specific excited state is
\begin{equation}
    \begin{split}
        a_n&=\braket{\psi_n(\llambda+\d\llambda)|\iota(\llambda)}\approx \d\lambda^\mu\braket{\partial_\mu \psi_n(\llambda)|\iota(\llambda} \\
        &= -\d\lambda^\mu\braket{\psi_n(\llambda)|\partial_\mu|\iota(\llambda)}\equiv -\d\lambda^\mu\braket{n|\partial_\mu|\iota},
    \end{split}
\end{equation}
where we introduced shortend notation for eigenstates of the Hamiltonian $\H_0$. If we introduce the \emph{gauge potential}, a.k.a \emph{calibration potential} as
\begin{equation}
    \AA_\mu\equiv i\hbar \partial_{\mu}
\end{equation}
and rescale units to $\hbar=1$, as we will use further on, we get
\begin{equation}
   a_n=\sum_\mu i\braket{n|\AA_\mu |\iota}\d\lambda^\mu,
\end{equation}
which has meaning of matrix elements of the gauge potential. Probability of the excitation i.e. transition to any state $n>0$ is then
\begin{equation}
    \begin{split}
        \sum_{n\neq 0}|a_n|^2&=  \sum_{n\neq 0} \d \lambda^\mu \d \lambda^\nu\braket{\iota|\AA_\mu|n}\braket{n|\AA_\nu|\iota}+\O(|\d \lambda^3|) = \d \lambda^\mu \d \lambda^\nu\braket{\iota|\AA_\mu \AA_\nu|\iota}_c\\
        &= \d \lambda^\mu \d \lambda^\nu\chi_{\mu\nu}+\O(|\d \lambda^3|)=\d s^2+\O(|\d \lambda^3|),
    \end{split}
\end{equation}
where we defined \emph{connected correlation function}, or \emph{covariance}
\begin{equation}
    \braket{\iota|\AA_\mu\AA_\nu|\iota}_c\equiv \braket{\iota|\AA_\mu\AA_\nu|\iota} - \braket{\iota|\AA_\mu|\iota}\braket{\iota|\AA_\nu|\iota}.
    \label{eq:covariance}
\end{equation}
If we leave out $\hbar$, we have the \emph{geometric tensor}\footnote{sometimes defined directly as the expression in eq. \ref{eq:covariance}}
\begin{equation}
    \chi_{\mu\nu}\equiv \braket{\partial_\mu \iota|\partial_\nu \iota}_c = \braket{\partial_\mu \iota|\partial_\nu \iota} - \braket{\partial_\mu \iota|\iota}\braket{\iota|\partial_\nu \iota},
\end{equation}
where $\ket{\partial_\nu \iota}\equiv\partial_\nu \ket{ \iota}$. Because $\chi$ is Hermitian ($\chi_{\mu\nu}=\chi^*_{\nu\mu}$), only the symmetric part adds up to the distance between states 
\begin{equation}
    \d s^2= g_{\mu\nu}\d \lambda^\mu \lambda^\nu= \chi_{\mu\nu}\d \lambda^\mu \lambda^\nu.
\end{equation}
 and only the symmetric part determines the distance between the states. Therefore it's practical to decompose it as
\begin{equation}
    \chi_{\mu\nu} \equiv g_{\mu\nu} - \textcolor{red}{i\frac{1}{2}} \nu_{\mu\nu},
\end{equation}
where the \emph{Fubini-Study tensor}, as it's called, is
\begin{equation}
    g_{\mu\nu} = \frac{\chi_{\mu\nu}+\chi_{\nu\mu}}{2} = \Re\braket{\partial_\mu i|\partial_\nu i}_c = \textcolor{red}{\Re \sum_{i\neq j}\frac{\braket{\iota|\pder{\H}{\lambda^\mu}|j}\braket{j|\pder{\H}{\lambda^\nu}|\iota}}{(E_i-E_j)^2}},
    \label{eq:geom.tensorREdefinition}
\end{equation}
and the \emph{curvature tensor} a.k.a. \emph{Berry curvature} is
\begin{equation}
    \begin{split}
        \nu_{\mu\nu} = 2 i(\chi_{\mu\nu}-\chi_{\nu\mu})&= \Im\braket{\iota|[\overleftarrow{\partial}_\nu,\partial_\mu]|\iota}_c \\
        &= -2 \Im \sum_{i\neq j}\frac{\braket{\iota|\pder{\H}{lambda^\mu}|j}\braket{j|\pder{\H}{lambda^\nu}|\iota}}{(E_i-E_j)^2},
    \end{split}
\end{equation}
where $\overleftarrow{\partial}_\nu$ is the derivative of the covector on the left.

Fubini-Study tensor can be seen as the Pull-back of the elements of the full Hilbert space to $\M$. 


Next we define \emph{Berry connection}
\begin{equation}
    A_\mu\equiv \braket{\iota|\AA_\mu|\iota},
\end{equation}
which empovers us to write
\begin{equation}
    \nu_{\mu\nu} = \partial_\mu A_\nu-\partial_\nu A_\mu
\end{equation}
and \emph{Berry phase}
    \footnote{
        The reasonability of this definition can be seen, if we assume the ground state of a free particle
            $\braket{\bm{x}|\iota}=\iota(\bm{x},\llambda)= |\iota(\bm{x})|e^{i\phi(\llambda)}$,
        then the Berry connection is
        \begin{equation}
            A_\mu=-\int \d \bm{x}|\iota|^2\partial_\mu \phi = -\partial_\mu \phi
        \end{equation} 
        and Berry phase
        \begin{equation}
            \phi_B=\oint_\mathcal{C} \partial_\mu \phi \d \lambda^\mu,
        \end{equation}
        which represents total phase accumulated by the wavefunction. It is really the analogy for Berry phase in classical mechanics, which for example for the Faucolt pendulum on one trip around the Sun makes $\phi_B=2\pi$
    }
\begin{equation}
    \phi_B\equiv-\oint_\mathcal{C} A_\mu \d \lambda^\mu=\int_\mathcal{S} F_{\mu\nu}\d \lambda^\mu \wedge \d\lambda^nu,
\end{equation}
where we used the Stokes theorem defining, that the curve $\mathcal{C}$ surrounds some area $\mathcal{S}$.

Wave-functions are elements of the tangent bundle $\TT\in \M$, the gauge potentials are affine connections defining the parallel transport. Covariant derivative is
\begin{equation}
    D_\mu=\partial_\mu+\frac{i}{\hbar}\AA_\mu,
\end{equation}
which yields $D_\mu\ket{\psi_n}=0$ for every eigenstate, which encloses the circle and \textcolor{red}{justifies our initial choise for the distance on $\M$.}

\chapter*{Conclusion}
\addcontentsline{toc}{chapter}{Conclusion}
This thesis presents the theory of quantum state driving in finite Hamiltonian systems. The well known claims were reformulated into more rigorous theorems and definitions. The playground in the form of fiber space was constructed, and the geometry of energy states reformulated on it. The idea of "advantageous driving" with high fidelity was scrutinized, especially as adiabatic, close-adiabatic and counter-diabatic driving. The mathematical approach to the thesis has potential to reduce the barrier for any mathematically based scientist when trying to approach this theory. 

The correspondence to a damped harmonic oscillator in the fidelity behavior was shown on a simple two-level system. In the case of geodesic driving, the fidelity oscillates with a constant frequency and periodically becomes one. For driving along straight line in a parametric space, the fidelity has essentially two regimes — fast transport regime, described by the semiclassical Landau-Zener formula, and close adiabatic regime, described by APT. In both models, the fidelity decreases with the difference between energy levels. Decrease in fidelity means state excitation, which leads to damped oscillations in fidelity.

In the LMG model, the ground state manifold was analyzed. The Riemannian manifold characteristics were calculated along with their implications. These are the ground state manifold geodesics and the coordinates of diabolic points in the parametric space, dependent on the Hamiltonian dimension. In three dimensions, this was done analytically, proving the existence of diabolic points in LMG model. For higher dimensions, numerical methods were used. The transport using quenches was numerically demonstrated, showing the transformation to adiabatic transport by shortening the quenches to zero.

Many questions still lay unsolved, and some were newly opened. For example: "What are the possibilities for quantum quench transport?". "What is the correspondence of energy variance and driving fidelity?". Further on, from the LMG model, the proposed analytical formula for diabolic points coordinates remains to be proven analytically, along with the number of these points.

This thesis provides a significant amount of numerical analysis on two quantum models, and offers a new way of looking at the theory of the quantum state driving. Hopefully, this will serve for future discoveries in this area of physics. The state manifold analyses might lead the search for better fidelity protocols, or maybe the geodesics will be found to be somewhat "most stable protocols" for counter-diabatic driving. Either way, the possibilities are yet open and not known.

%%% Bibliography
\include{bibliography}

%%% Figures used in the thesis (consider if this is needed)
\listoffigures

%%% Tables used in the thesis (consider if this is needed)
%%% In mathematical theses, it could be better to move the list of tables to the beginning of the thesis.
%\listoftables

%%% Abbreviations used in the thesis, if any, including their explanation
%%% In mathematical theses, it could be better to move the list of abbreviations to the beginning of the thesis.
%\chapwithtoc{List of Abbreviations}

%%% Attachments to the master thesis, if any. Each attachment must be
%%% referred to at least once from the text of the thesis. Attachments
%%% are numbered.
%%%
%%% The printed version should preferably contain attachments, which can be
%%% read (additional tables and charts, supplementary text, examples of
%%% program output, etc.). The electronic version is more suited for attachments
%%% which will likely be used in an electronic form rather than read (program
%%% source code, data files, interactive charts, etc.). Electronic attachments
%%% should be uploaded to SIS and optionally also included in the thesis on a~CD/DVD.
%%% Allowed file formats are specified in provision of the rector no. 72/2017.
\appendix
\chapter{Attachments}

\section{First Attachment}

\openright
\end{document}
