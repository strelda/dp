\chapter{Mathematical introduction}
The modern approach to the closed system dynamics is using differential geometry formalism. Before we get to the quantum mechanics itself, lets define the formalism and recapitulate some definitions of this branch of mathematics. More detailed notes can be found for example in \citep{fecko}.

Let's have a manifold $\M$ and curves 
$$\gamma:\R \overset{open}{\supset} I \rightarrow \M \qquad \xi\mapsto \gamma(\xi).$$ 
The space of functions is $\F(\M)\equiv\{f:\M\rightarrow \R\}$, where 
$$f:\M\rightarrow U\overset{open}{\subset} \R \qquad x\mapsto f(x).$$
To define \emph{vectors} on $\M$, we need to make sense of the \emph{direction}. It is defined using curves satisfying 
$$\gamma_1(0)=\gamma_2(0)\equiv P$$
$$\der{}{t}x^i(\gamma_1(t))\big|_{t=0}=\der{}{t}x^i(\gamma_2(t))\big|_{t=0}.$$
Taking the equivalence class created by those two rules, sometimes noted as $[\gamma]=v$, we have element of the tangent space to $\M$. We will use standard notation for the tangent space to $\M$ in some point $xP$ as $\T_P\M$ and contangent space as $\T^*_P\M$. Unifying all those spaces over all $x$ we get tangent and cotangent bundle, $\TT\M$ and $\TT^*\M$ respective. To generalize this notation to higher tensors, we denote $\T_P\M\in \TT^1\M$, $\T^*_P\M\in \TT_1\M$, thus the space of $p-$times contravariant and $q-$times covariant tensors is denoted $\TT^p_q\M$.

Using the congruence of the curves on $\M$, the expression 
\begin{equation}
    \der{}{\xi}f\circ \gamma(\xi)\Big|_{\xi=0}
\end{equation}
has a good meaning and we can define the \emph{derivative} in some $P\in\M$ as
\begin{equation}
    \bm v: \F(\M)\rightarrow \R \qquad f\mapsto \bm v[f]\equiv \frac{\d f(\gamma(\xi))}{\d \xi}\Big|_P \equiv \partial_\xi\Big|_P f .
\end{equation}
It holds, that $\bm v\in \T_P\M$ and can be expressed as the \emph{derivative in direction},\footnote{
        The direction itself is usually denoted as
        \begin{equation}
            \frac{\D}{\d\bm\alpha}\gamma(\xi),
        \end{equation}
        where the big D notation is used to point out that it's not a classical derivative, but it maps curves to some entirely new space of directions.
    } 
which can be understood in coordinates as
\begin{equation}
    \bm v[f] = \der{}{\bm v} f\circ \gamma(\xi)\Big|_{\xi=0}=v^\mu\der{}{x^\mu} f(\bm x)\Big|_{P}.
\end{equation}
The directionnal derivative will be denoted 
$$\nabla_v$$
and in basis $\bm e_i \equiv \partial/\partial x^i$ we will denote 
$$\nabla=(\bm e_x, \bm e_y,\bm e_z)$$



To get some physical application, we need to define one strong structure on manifolds -- differentiable metric tensor $g_{\mu\nu}\in\TT^0_2\M$ -- so the covariant derivatives and parallel transport are well-defined everywhere. 


\section{Pull-back and push forward}
Push-forward and pull-back are used to transport vectors and covectors between manifolds. Let's have two manifolds $\M$, $\mathcal{N}$, a smooth mapping $\phi$ and functions $f,\tilde f$ such that
\begin{align*}
    \phi&:\M\rightarrow \mathcal{N}\qquad x\mapsto \phi x\\
    \tilde f&:\mathcal{N}\rightarrow \R 
\end{align*}
\emph{Pull-back of the function} then defines a new function $
f:\M\rightarrow \R $ as
$$\phi^*:\F \mathcal{N}\rightarrow \F\M \qquad  \tilde f\mapsto f=(\phi^*\tilde f)(x)\equiv \phi^*\tilde f(x) =\tilde f(\phi x).$$
\emph{Push-forward of a vector} is defined as
$$\phi_*: \T_x\M \rightarrow \T_{\phi x}\mathcal N\qquad \phi_* 
\Der{\gamma(\xi)}{\xi}\Big|_x=\Der{\phi \gamma(\xi)}{\xi}\Big|_x$$
and \emph{pull-back of a covector} $\bm\tilde\alpha\in \T_{\phi x}\mathcal N$ is
$$\phi^*: \T_{\phi x}\mathcal N\rightarrow \T_x\M  \qquad (\phi^*\bm \tilde\alpha)_\mu v^\mu\big|_x= \tilde\alpha_\mu (\phi_* \bm v)^\mu\big|_{\phi x}.$$
If $\phi$ has a smooth inversion, i.e. it is a dippheomorphism, we can define pull-back of vectors as
\begin{equation}
    \phi^*=\phi_*^{-1}
\end{equation}
and push-forward of covectors
\begin{equation}
    \phi_*=(\phi^{-1})^*
\end{equation}


\section{Flow}
% Flow is
% \begin{equation}
%     \phi_*\in \textrm{Diff}\M
% \end{equation}


\section{Covariant derivative and parallel transport}
Covariant derivative is generally\dots
Metric covariant derivative is\dots

Parallel transport of vector $\bm v\in \T_p \M$ will be denoted $\Par_\gamma \bm v\in \T_{}$

Affine connection can be expressed as
\begin{equation}
    \Gamma^{\alpha}_{\mu\nu} = \frac{1}{2}g^{\alpha \beta}\left(g_{\beta\mu,\nu}+g_{\nu\beta,\mu}-g_{\mu\nu,\beta}\right),
\end{equation}
where we used comma notation for the coordinate derivative.
The covariant derivative of $\bm a\in \T_P\M$ is then defined
\begin{equation}
    \Der{a^\mu}{x^\nu}=a_{\;,\nu}^\mu-\Gamma^\mu_{\alpha\beta} x^\alpha a^\beta 
\end{equation}
and for $\bm\alpha\in \T_P^*\M$ it is
\begin{equation}
    \Der{\alpha_\mu}{x^\nu}=\alpha_{\mu,\nu}-\Gamma^\alpha_{\mu\beta}x^\beta \alpha_\alpha 
\end{equation}

The vector $v\in \T_P\M$ is said to be parallel transported along curve $\gamma(\lambda)$, if it's covariant derivative
\begin{equation}
    \Der{v^\mu}{\xi}=0
\end{equation}
vanishes along $\gamma$.

\section{Antisymmetric tensors and wedge product}
p-form $A\in \TT_p \M$ is called \emph{antisymmetric}, if changing the order of the indices has impact only on the sign, symbolically
$$A_{i_1\dots i_p} = \sign(\sigma)A_{i_{\sigma_1}\dots i_{\sigma_p}},$$
where $\sigma$ is some permutation.\emph{Antisymmetrisation} is defined as a normalized sum over all permutation
\begin{equation}
    A^{[i_1\dots i_p]}\equiv \frac{1}{p!}\sum_\sigma A^{[i_{\sigma_1}\dots i_{\sigma_p}]}. 
\end{equation}

The \emph{wedge product} of $A\in \TT_p \M$ and $B\in \TT_q \M$ is antisymmetrisation of the tensor product in the sense
\begin{equation}
    A\wedge B\equiv \frac{(p+q)!}{p!q!} A^{[i_1\dots i_p}\otimes B^{i_1\dots i_q]}
\end{equation}
