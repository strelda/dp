%%%
%%%  Template to prepare a defense of Bc./Mgr./... thesis
%%%  to be presented at MFF
%%%  (unofficial)
%%%
%%%  AUTHOR:  Arnošt Komárek
%%%           Department of Probability and Mathematical Statistics
%%%           Faculty of Mathematics and Physics, Charles University in Prague
%%%
%%%  LOG:    20150505  created by modification of some previous personal presentations
%%%          20170522  update related to new MFF logo
%%%  
%%%  ===========================================================================
\documentclass[c, 10pt]{beamer}


%%%%% Package needed if some accented letters in the presentation
%%%%% -------------------------------------------------------------
\usepackage[utf8]{inputenc}


%%%%% Most beamer settings and other LaTeX commands
%%%%% are provided in the file below
%%%%% -----------------------------------------------------
%%%
%%%  Style for MFF related presentations
%%%  (unofficial)
%%%
%%%  AUTHOR:  Arnošt Komárek
%%%           Department of Probability and Mathematical Statistics
%%%           Faculty of Mathematics and Physics, Charles University in Prague
%%%
%%%  LOG:    20150430  created by modification of some previous personal presentations
%%%          20170522  modified (new MFF logo)
%%%  
%%%  ===========================================================================

%%%%% If to determine whether Czech or English version
%%%%% of a presentation is to be produced
%%%%% default = Czech
%%%%% ------------------------------------------------------
\newif\ifCZversion
\CZversiontrue

\ifCZversion\else\renewcommand{\uv}[1]{``#1''}\fi


%%%%% Included packages
%%%%% ----------------------------------------------------------
\usepackage{helvet}                       % font
\usepackage{amsmath, amssymb}
\usepackage{delarray}
\usepackage{multicol}
\usepackage{graphicx, fancybox}
\usepackage{psfrag}
\usepackage{fancyvrb}
\usepackage{eurosym}
\usepackage{bbding}
%\usepackage{marvosym}
\usepackage{wasysym}
\usepackage[czech]{babel}

\usepackage[pdftex]{bookmark}


%%%%% Some LaTeX commands
%%%%% ----------------------------------------------------------
\renewcommand{\arraystretch}{1.2}


%%%%% Some colors and related commands
%%%%% ----------------------------------------------------------

  %% Pantone 186 = "official" red of MFF
\definecolor{redmff}{rgb}{0.7892720,0.0651341,0.1455939}          

  %% First level color = black
\definecolor{colOne}{rgb}{0,0,0}
\newcommand{\tOne}[1]{\textcolor{colOne}{#1}}
\newcommand{\tOneb}[1]{\textcolor{colOne}{\textbf{#1}}}
\newcommand{\tOnei}[1]{\textcolor{colOne}{\textit{#1}}}

  %% Second level color = quite dark blue
\definecolor{colTwo}{rgb}{0,0,0.3}                
\newcommand{\tTwo}[1]{\textcolor{colTwo}{#1}}
\newcommand{\tTwob}[1]{\textcolor{colTwo}{\textbf{#1}}}
\newcommand{\tTwoi}[1]{\textcolor{colTwo}{\textit{#1}}}

  %% Third level color = something between blue3 and blue4
\definecolor{colThree}{rgb}{0,0,0.7}              
\newcommand{\tThree}[1]{\textcolor{colThree}{#1}}
\newcommand{\tThreeb}[1]{\textcolor{colThree}{\textbf{#1}}}
\newcommand{\tThreei}[1]{\textcolor{colThree}{\textit{#1}}}

  %% Color to alert (highlight) = MFF red
\definecolor{alertCol}{rgb}{0.7892720,0.0651341,0.1455939}
\newcommand{\tal}[1]{\alert{#1}}
\newcommand{\talb}[1]{\textbf{\alert{#1}}}
\newcommand{\tali}[1]{\textit{\alert{#1}}}

  %% White
\newcommand{\tw}[1]{\textcolor{white}{#1}}
\newcommand{\twb}[1]{\textcolor{white}{\textbf{#1}}}
\newcommand{\twi}[1]{\textcolor{white}{\textit{#1}}}

  %% Some other colors used somewhere
\definecolor{semiwhite}{gray}{0.98}
\definecolor{mffgray}{gray}{0.95}
\definecolor{semiblack}{gray}{0.5}
\definecolor{rred}{rgb}{0.5,0,0}


%%%%% Beamer stuff
%%%%% ----------------------------------------------------------
\mode<presentation> {

    %%% General theme of a presentation (some pre-specified themes)
    %%% ------------------------------------------------------------------
    %\usetheme{CambridgeUS}
    \usetheme{Warsaw}

    %%% Color schemes that will be used unless redefined below
    %%% ------------------------------------------------------------------
    %\usecolortheme{wolverine}
    \usecolortheme{beaver}


    %%% Color schemes for different elements of a presentation
    %%% ---------------------------------------------------------

      %%% Uncomment the two rows below if some background file (layout) is to be used
      %%% on all slides
    %\usebackgroundtemplate{\includegraphics[width=\paperwidth, height=\paperheight]{./FigureLayout/BackgroundRotunda}}
    %\setbeamercolor{background canvas}{bg=}                                                             

      %%% Setting for "standard" slides
      %%% - must be commented if background file used
    \setbeamertemplate{background canvas}[vertical shading][bottom=semiwhite, top=white, middle=semiwhite!50!white]
    \setbeamercolor{background canvas}{bg=white}                           %% NO EFFECT WHEN \setbeamertemplate{background canvas} WAS USED

      %%% Colors of frametitle, normal, alerted and math texts
    \setbeamercolor{frametitle}{fg=white, bg=redmff}
    \setbeamercolor{normal text}{fg=black}
    \setbeamercolor{alerted text}{fg=redmff}
    \setbeamercolor{math text inlined}{fg=colTwo}
    \setbeamercolor{math text displayed}{fg=colTwo}

      %%% Colors for some special boxes defined below
    \setbeamercolor{mffboxcol}{fg=colTwo, bg=mffgray}
    \setbeamercolor{mffboxcolupper}{fg=white, bg=redmff}

    %%% Set serif font (patkové písmo) in mathematics
    %%% ------------------------------------------------
    %\usefonttheme[onlymath]{serif}

    %%% Itemize style (rectangles instead of bullets)
    %%% ------------------------------------------------
    %\useinnertheme{rectangles}

    %%% Default content of a header of each slide
    %%% (currently nothing)
    %%% ------------------------------------------------
    \setbeamertemplate{headline}{}

    %%% Default content of a foot of each slide
    %%% ------------------------------------------------
    %\setbeamercolor{page number in head/foot}{fg=black, bg=white}
    \setbeamertemplate{footline}{
      \hspace*{2.5em}
      \begin{beamercolorbox}{section in head/foot}
      \vskip1pt
      \tOne{\rule{\textwidth}{1pt}}
      \vskip2pt

      %%% Foot containing (a) page number/total number of pages, (b) name, (c) short title of presentation
      {\small \tOne{\insertframenumber}\textcolor{semiblack}{/\inserttotalframenumber}}\hspace{1em}
      \tOne{\footnotesize \insertauthor}\hfill
      \textcolor{redmff}{\footnotesize\insertshorttitle\hspace{3em}}

      %%% Foot containing (a) page number/total number of pages, (b) name, (c) short section title
      %{\small \tOne{\insertframenumber}\textcolor{semiblack}{/\inserttotalframenumber}}\hspace{1em}
      %\tOne{\footnotesize \insertauthor}\hfill
      %\textcolor{redmff}{\footnotesize\thesection. \insertsection\hspace{3em}}
    
      \hspace*{3.5em}
      \vskip3pt
      \end{beamercolorbox}
    }

    %%% Title page
    %%% -------------------
    \setbeamertemplate{title page}{
      %\vspace*{-0.5em}
      \begin{center}
      \textcolor{black}{\normalsize\bfseries\rmfamily \insertinstitute}

      \vspace{1em}
        
      \ifCZversion
        \includegraphics[width=0.8\textwidth]{./FigureLayout/mff_cz_color}
      \else
        \includegraphics[width=0.8\textwidth]{./FigureLayout/mff_en_color}
      \fi

      \medskip
      \noindent\textcolor{redmff}{\rule{\textwidth}{2pt}}
  
      \bigskip
      \textcolor{black}{\normalsize\bfseries \insertauthor} \\[0.5ex]

      \bigskip
      \textcolor{redmff}{\Large\bfseries \inserttitle}

      \medskip
      \textcolor{redmff}{\large\insertsubtitle}

      \medskip
      \noindent\textcolor{redmff}{\rule{\textwidth}{2pt}}
      
      \medskip
      \textcolor{colTwo}{\small \insertdate}
      \end{center}
    }    

    %%% Switch-off/on foot with navigation symbols    
    %%% -----------------------------------------------
    % \setbeamertemplate{navigation symbols}{}
    \usenavigationsymbolstemplate{}

    %%% Left and right margin
    %%% --------------------------------
    %\setbeamersize{text margin left=1cm}
    %\setbeamersize{text margin right=1cm}


    %%% Use of a logo on each slide
    %%% (not really recommended, so commented)
    %\logo{\includegraphics[height=1.5cm, width=1.5cm]{./FigureLayout/mff_logo}}
}


%%%%% Commands to produce slides at the beginning of each section
%%%%% --------------------------------------------------------------
\renewcommand{\thesection}{\arabic{section}}
\newcommand{\framesection}{
  \begin{frame}%[plain]

  \vspace*{2em}
  \begin{center}\Large
  \ifCZversion Oddíl \thesection \else Section \thesection \fi
  \end{center}

  \begin{center}\color{rred}\Large
  \insertsection
  \end{center}

  \end{frame}
}


%%%%% Style for software related stuff
%%%%% ----------------------------------------------------------
\newcommand{\Rko}{\includegraphics[width=5.094mm, height=3.876mm]{./FigureLayout/Rlogo}}
\newcommand{\Rfun}[1]{\textcolor{redmff}{\texttt{#1}}}

\DefineVerbatimEnvironment{Rin}{Verbatim}{formatcom=\color{redmff}, fontsize=\scriptsize, frame=single, framerule=1pt, framesep=2pt}
\DefineVerbatimEnvironment{Rout}{Verbatim}{formatcom=\color{blue}, fontsize=\scriptsize, frame=single, framerule=1pt, framesep=2pt}


%%%%% Boxes
%%%%% ----------------------------------------------------------
\newcommand{\mffbox}[2][0pt]{%
  \begin{beamercolorbox}[center, sep=#1, rounded=true, shadow=true]{mffboxcol}
  #2
  \end{beamercolorbox}
}

\newcommand{\mffboxTitle}[2]{%
  \begin{beamerboxesrounded}[lower=mffboxcol, upper=mffboxcolupper, shadow=true]{#1}
  #2
  \end{beamerboxesrounded}
}


%%%%% Some constructions for displayed math
%%%%% ----------------------------------------------------------
\newcommand{\dmath}[2][-1.4em]{%
  \begin{beamercolorbox}[center, sep=#1, rounded=true, shadow=true]{mffboxcol}
  \begin{displaymath}
  #2
  \end{displaymath}
  \end{beamercolorbox}
}

\newcommand{\dalign}[2][-1.4em]{%
  \begin{beamercolorbox}[center, sep=#1, rounded=true, shadow=true]{mffboxcol}
  \begin{align*}
  #2
  \end{align*}
  \end{beamercolorbox}
}

\newcommand{\dgather}[2][-1.4em]{%
  \begin{beamercolorbox}[center, sep=#1, rounded=true, shadow=true]{mffboxcol}
  \begin{gather*}
  #2
  \end{gather*}
  \end{beamercolorbox}
}



%%%%% \ifCZversion is defined inside MFF_Present.sty
%%%%% to distinguish between Czech and English presentations
%%%%% ------------------------------------------------------
\CZversiontrue       %% for presentations in Czech (Slovak)
%\CZversionfalse      %% for presentations in English


%%%%% Uncomment appropriate choice below if you wish to create
%%%%% notes for audience having 2 or 4 slides on each (A4) page.
%%%%% -------------------------------------------------------------
\usepackage{pgfpages}
%\pgfpagesuselayout{4 on 1}[a4paper, landscape, border shrink=5mm]
%\pgfpagesuselayout{2 on 1}[a4paper, border shrink=5mm]


%%%%% Basic settings of the document
%%%%% (will be automatically used to create a title page, foots etc.)
%%%%% --------------------------------------------------------------------

  %%% Main title
  %%% - short and long version
  %%%   --> will appear on place where \inserttitle and \insertshorttitle commands used
  %%%   --> if the full title is short enough, both short and long versions might be the same
\title[Kratší název]{%                       
       Plný název bakalářské práce}

  %%% Subtitle (comment it if you do not want to have it)
  %%%   --> will appear on place where \insertsubtitle and \insertshortsubtitle commands used
\subtitle[]{Obhajoba bakalářské práce}

  %%% Author
  %%% - as "short" version, link to the author's webpage is used
  %%%   (e.g., e-mail is also a useful alternative)
  %%%   --> will appear on places where \insertauthor and \insertshortauthor commands used
\author[http://msekce.karlin.mff.cuni.cz/\~{}komarek]{%
        Arnošt Komárek}

  %%% Author's affiliation
  %%% - can be fully commented for defense presentation
  %%%   --> will appear on places where \insertinstitute and \insertshortinstitute commands used
\institute[KPMS]{%
           Katedra pravděpodobnosti a~matematické statistiky}

  %%% Date of presentation
  %%% - replace it by real date in case of a defense presentation
  %%%   --> will appear on places where \insertdate and \insertshortdate commands used
\date[22.6.2017]{%
      22. června 2017}


\begin{document}

%%%%% Title slide
%%%%% =====================================================================================
\frame[plain]{\titlepage}


%%%%% Fictitious introductory section
%%%%% =====================================================================================
\section{Úvod}
%\framesection{}     %%% Uncomment it to get a special slide with the section title
                     %%% - not really needed for a presentation lasting 10 minutes

  %%%%% Slide
  % ----------------------------------------------------------------------------------------
\begin{frame}\frametitle{Smysl prezentace}
\framesubtitle{Bakalářská práce}

\begin{itemize}\itemsep=1em
\item Stručně seznámit komisi s~obsahem práce.
\item Vysvětlit \alert{hlavní} myšlenky.
\item Vysvětlit, v~čem spočívá \alert{hlavní přínos studenta} k~dané problematice.
  \begin{itemize}\color{colTwo}\itemsep=1ex
  \item Toto je nejdůležitější součást prezentace.
  \item Komisi, jejíž většina členů nečetla podrobně celou bakalářskou práci,
    je potřeba přesvědčit, že se nejedná o~překlad jakéhosi anglického textu 
    do češtiny/slovenštiny.
  \item V~rámci prezentace je vhodné zdůraznit, jaké (matematické) problémy musel
    autor bakalářské práce samostatně vyřešit.
  \end{itemize}
\item Smyslem obhajoby bakalářské práce \alert{není} naučit posluchače 
  matematiku obsaženou v~práci. Nejedná se o~obdobu klasické přednášky!
\end{itemize}
\end{frame}

  %%%%% Slide
  % ----------------------------------------------------------------------------------------
\begin{frame}\frametitle{Základní zásady prezentace}
\framesubtitle{Bakalářská práce}

\begin{itemize}\itemsep=1em
\item Celková doba prezentace by neměla přesáhnout \alert{10 minut!}
\item Méně je někdy více! Nicméně všeho s~mírou. Prezentace končící po pěti minutách též nebude působit příliš dobře.
\end{itemize}
\end{frame}

  %%%%% Slide
  % ----------------------------------------------------------------------------------------
\begin{frame}\frametitle{Další zásady prezentace}
\framesubtitle{Bakalářská práce}

\begin{itemize}\itemsep=1em
\item V~prezentaci (na slidech) by se (až na výjimky) neměl objevit souvislý text.
  \begin{itemize}\color{colTwo}\itemsep=1ex
  \item Slidy obsahují pouze klíčovou část informace, kterou má prezentace posluchačům předat.
  \item Předpokládá se, že slidy jsou doplňovány mluveným projevem, který je tvořen
   \alert{souvislými} větami přednesenými \alert{spisovným} jazykem. 
  \end{itemize}
\item Nepůsobí dobře, je-li mluvený projev předčítán z~papíru drženého v~ruce
   \textit{(obhajoba není projevem na stranickém kongresu/sjezdu).}
  \begin{itemize}\color{colTwo}
  \item Hlavní osnova prezentace by měla být zřejmá ze slidů, zbytek by měl být uložen
    v~hlavě prezentujícího. 
  \end{itemize}
\item Taktéž však nepůsobí úplně dobře, připomíná-li prezentace doslovný přednes naučeného textu
   \textit{(obhajoba není soutěží v~recitaci).}
  \begin{itemize}\color{colTwo}
  \item Mluvený projev by měl být přirozený. S~jistou nervozitou je počítáno.
  \end{itemize}
\end{itemize}
\end{frame}


%%%%% Fictitious main section
%%%%% =====================================================================================
\section{Hlavní část}
%\framesection{}     %%% Uncomment it to get a special slide with the section title
                     %%% - not really needed for a presentation lasting 10 minutes

  %%%%% Slide
  % ----------------------------------------------------------------------------------------
\begin{frame}\frametitle{Zásadní poznatek}

%\begin{beamercolorbox}[center, sep=-1em, rounded=true, shadow=true]{mffboxcol}
\begin{displaymath}
F(x) = \mathsf{P}(X \leq x),\qquad x\in\mathbb{R}.
\end{displaymath}
%\end{beamercolorbox}
\begin{itemize}
\item Toto je \alert{zprava} spojitá verze distribuční funkce.
\end{itemize}
\end{frame}


%%%%% Fictitious final section
%%%%% ===================================================================================
\section{Závěr}
%\framesection{Závěr}     %%% Uncomment it to get a special slide with the section title
                          %%% - not really needed for a presentation lasting 10 minutes

  %%%%% Slide
  % ----------------------------------------------------------------------------------------
\begin{frame}[plain]         %%% plain = no title etc.

%%% Rámeček
\begin{beamercolorbox}[center, sep=2pt, rounded=true, shadow=true]{mffboxcol}
\LARGE\bfseries \alert{Děkuji za pozornost!}
\end{beamercolorbox}

\vspace{5em}
\begin{beamercolorbox}[center, sep=2pt, rounded=true, shadow=true]{mffboxcol}
Za ochotu a~čas mně věnovaný při přípravě této bakalářské práce děkuji též svému vedoucímu \alert{prof. Janu Jakubovi}.
\end{beamercolorbox}
\end{frame}


%%%%% Fictitious reactions to comments of the reviewer
%%%%% ===================================================================================
\section{Reakce na připomínky oponenta}
%\framesection{}

  %%%%% Slide
  % ----------------------------------------------------------------------------------------
\begin{frame}\frametitle{Připomínky oponenta}

\begin{itemize}\itemsep=1ex
\item Uvedl-li oponent ve svém posudku zásadnější připomínky nebo dotazy, je vhodné 
  si připravit relevantní odpovědi písemnou formou.
\item Tyto se zařadí na konec prezentace (za poděkování) a~použijí se v~případě,
  že je během obhajoby vyžadována podrobnější reakce na tu kterou připomínku.
\item Písemné odpovědi není nutné připravovat pro formální připomínky, resp. pro
  připomínky/dotazy, které lze odpovědět/vysvětlit jednou větou.
\end{itemize}
\end{frame}


%%%%% Illustrations of use of some beamer capabilities
%%%%% ===================================================================================
\section{Ilustrace použití {\LaTeX} balíčku \texttt{beamer}}
\framesection{}

  %%%%% Slide
  % ----------------------------------------------------------------------------------------
\begin{frame}\frametitle{Vysazený vzorec v rámečku}
\framesubtitle{Pomocí příkazu \texttt{dmath} definovaného v~\texttt{MFF\_{}Present.sty}}

\dmath{
  F(x) = \mathsf{P}(X \leq x),\qquad x\in\mathbb{R}.
}
\begin{itemize}
\item Toto je \alert{zprava} spojitá verze distribuční funkce.
\end{itemize}
\end{frame}

  %%%%% Slide
  % ----------------------------------------------------------------------------------------
\begin{frame}\frametitle{Vysazený vzorec v rámečku}
\framesubtitle{Pomocí příkazu \texttt{dmath} definovaného v~\texttt{MFF\_{}Present.sty}, menší okraje}

\dmath[-1.6em]{
  F(x) = \mathsf{P}(X \leq x),\qquad x\in\mathbb{R}.
}
\begin{itemize}
\item Toto je \alert{zprava} spojitá verze distribuční funkce.
\end{itemize}
\end{frame}

  %%%%% Slide
  % ----------------------------------------------------------------------------------------
\begin{frame}\frametitle{Vysazené vzorce v rámečku (zarovnané)}
\framesubtitle{Pomocí příkazu \texttt{dalign} definovaného v~\texttt{MFF\_{}Present.sty}}

\dalign{
  F(t) &= \mathsf{P}(T \leq t),\qquad t > 0, \\[1ex]
  S(t) &= \mathsf{P}(T > t).
}
\begin{itemize}
\item Toto je \alert{zprava} spojitá verze distribuční funkce,
  resp. funkce přežití.
\end{itemize}
\end{frame}

  %%%%% Slide
  % ----------------------------------------------------------------------------------------
\begin{frame}\frametitle{Vysazené vzorce v rámečku (vycentrované)}
\framesubtitle{Pomocí příkazu \texttt{dgather} definovaného v~\texttt{MFF\_{}Present.sty}}

\dgather{
  F(t) = \mathsf{P}(T \leq t),\qquad t > 0, \\[1ex]
  S(t) = \mathsf{P}(T > t).
}
\begin{itemize}
\item Toto je \alert{zprava} spojitá verze distribuční funkce,
  resp. funkce přežití.
\end{itemize}
\end{frame}

  %%%%% Slide
  % ----------------------------------------------------------------------------------------
\begin{frame}\frametitle{Matematika v~titulku: $\tw{F(t) = \mathsf{P}(T \leq t)}$}

\begin{itemize}\itemsep=1em
\item Matematický text (v~dolarech) má nastaven svůj styl (zejména barvu).
\item Je-li matematika použita v~titulku slidu, je potřeba ji obarvit na
   standardní barvu použitou v~titulcích (zde bílá, pro kterou máme 
   ve stylovém souboru \texttt{MFF\_{}Present.sty} definován příkaz \texttt{tw}).
\end{itemize}

\end{frame}

  %%%%% Slide
  % ----------------------------------------------------------------------------------------
\begin{frame}\frametitle{Tučná a/nebo obarvená matematika}

\begin{itemize}\itemsep=1ex
\item Též v~rámci matematického textu můžeme zvýrazňovat (změnou barvy) nejdůležitější součásti vzorců:
  \dmath{
    \alert{F(t) = \mathsf{P}(T \leq t)}, \qquad t > 0.
  }

\item U~tučných symbolů získaných pomocí příkazu \texttt{boldsymbol} 
  je (bohužel) potřeba měnit barvu po jednom. U~tučných matematických fontů
  získaných pomocí \texttt{mathbf} toto potřeba není. Srovnej:
  \dgather{
    \tal{\widehat{\boldsymbol{\beta}} = \bigl(\mathbb{X}^\top\mathbb{X}\bigr)^{-1}\,\mathbb{X}^\top\mathbf{Y}}, \\[1ex]
    \tal{\widehat{\boldsymbol{\beta}} = \bigl(\mathbb{X}^\top\mathbb{X}\bigr)^{-1}\,\mathbb{X}^\top\boldsymbol{Y}}, \\[1ex]
    \tal{\widehat{\boldsymbol{\tal{\beta}}} = \bigl(\mathbb{X}^\top\mathbb{X}\bigr)^{-1}\,\mathbb{X}^\top\boldsymbol{\tal{Y}}}.
  }
  %%% \tal{} = \textcolor{alertCol}{}, see MFF_Present.sty
\end{itemize}

\end{frame}

  %%%%% Slide
  % ----------------------------------------------------------------------------------------
\begin{frame}\frametitle{Text v~barevném rámečku}
\framesubtitle{Pomocí příkazu \texttt{mffbox} definovaného v~\texttt{MFF\_{}Present.sty}}

\mffbox{
  Příliš žluťoučký kůň úpěl ďábelské ódy.
}

\vspace{2em}
Ještě jednou, nyní se změněnou velikostí okrajů:
\mffbox[1em]{
  Příliš žluťoučký kůň úpěl ďábelské ódy.
}

\end{frame}

  %%%%% Slide
  % ----------------------------------------------------------------------------------------
\begin{frame}\frametitle{Text v~barevném rámečku s~titulkem}
\framesubtitle{Pomocí příkazu \texttt{mffboxTitle} definovaného v~\texttt{MFF\_{}Present.sty}}

\mffboxTitle{Titulek}{
  Příliš žluťoučký kůň úpěl ďábelské ódy.
}

\vspace{2em}
Vhodné použít např. pro znění matematických vět:
\mffboxTitle{\textbf{Věta}. O~žlutém koni a~úrokové limitě}{
  Příliš žluťoučký kůň úpěl ďábelské ódy:
  \begin{displaymath}
  \lim_{n \to \infty} \biggl(1 + \frac{1}{n}\biggr)^n = \mathsf{e}.
  \end{displaymath}
}

\end{frame}

  %%%%% Slide
  % ----------------------------------------------------------------------------------------
\begin{frame}\frametitle{Odrážky a~jejich postupné odkrývání}

\begin{itemize}[<+->]\itemsep=1em
\item Odrážky se budou postupně odkrývat.
\item Obecně není dobré přehánět to s~efekty podobného typu, aby se prezentace nezvrhla
  v~rychlou změť postupně se objevujících částí textu.
\item Další řádek.
\item Ještě jeden řádek.
\end{itemize}

\end{frame}

  %%%%% Slide
  % ----------------------------------------------------------------------------------------
\begin{frame}\frametitle{Postupné odkrývání ještě jednou}
\framesubtitle{Příkaz \texttt{pause}}

\dgather{
  F(t) = \mathsf{P}(T \leq t),\qquad t > 0, \\[1ex]
  \mathbb{E}T = \int_0^\infty \bigl\{1 - F(t)\bigr\}\,\mathsf{d}t.
}

\pause
\vspace{2em}
\begin{itemize}
\item Platí pro náhodné veličiny, které jsou skoro jistě nezáporné a~mají konečnou střední hodnotu.
\end{itemize}

\end{frame}

  %%%%% Slide where fancyvrb environment used
  % ----------------------------------------------------------------------------------------
\begin{frame}[fragile]\frametitle{Ukázka kódu v~{\Rko}}   %%% [fragile] needed due to fancyvrb, otherwise LaTeX problem
\framesubtitle{Užití prostředí \texttt{Rin} a~\texttt{Rout} definovaných v~\texttt{MFF\_{}Present.sty}}

\alert{\bfseries Průměr}
\begin{Rin}
> mean(c(1, 2, 3, 4, 5))
\end{Rin}

\medskip
\begin{Rout}
[1] 3
\end{Rout}

\medskip
\alert{\bfseries Průměr menším písmem}
\begin{Rin}[fontsize=\scriptsize]
> mean(c(1, 2, 3, 4, 5))
\end{Rin}
\end{frame}

  %%%%% Slide
  % ----------------------------------------------------------------------------------------
{
\usebackgroundtemplate{\includegraphics[width = \paperwidth, height = \paperheight]{./FigureLayout/BackgroundRotunda}}
\begin{frame}\frametitle{Speciální pozadí na jednom slidu}
\framesubtitle{Formát hlavičky i~paty ponechán původní}

\begin{itemize}\itemsep=1em
\item \textcolor{rred}{Slide s~pozadím.}
\end{itemize}

\end{frame}
}

  %%%%% Slide
  % ----------------------------------------------------------------------------------------
{
\usebackgroundtemplate{\includegraphics[width = \paperwidth, height = \paperheight]{./FigureLayout/BackgroundRotunda}}
\begin{frame}[plain]\frametitle{Speciální pozadí na jednom slidu}
\framesubtitle{Pouze hlavička zůstala původní}

\begin{itemize}\itemsep=1em
\item \textcolor{rred}{Slide s~pozadím.}
\end{itemize}

\end{frame}
}

  %%%%% Slide
  % ----------------------------------------------------------------------------------------
{\usebackgroundtemplate{\includegraphics[width = \paperwidth, height = \paperheight]{./FigureLayout/BackgroundRotunda}}
\begin{frame}[plain]

\begin{itemize}\itemsep=1em
\item \textcolor{rred}{Slide s~pozadím (bez hlavičky).}
\end{itemize}

\end{frame}
}

  %%%%% Slide
  % ----------------------------------------------------------------------------------------
\begin{frame}\frametitle{{\LaTeX} balíček \texttt{beamer}}

\begin{itemize}
\item Mnoho dalších efektů vylepšujících (někdy) prezentaci lze
  nalézt v~dokumentaci {\LaTeX}ového balíku \alert{\texttt{beamer}}, např.

  \medskip
  \begin{flushleft}\footnotesize
  \tTwo{\texttt{http://ftp.cvut.cz/tex-archive/macros/latex/contrib/beamer/doc/}} \\
  \tTwo{\texttt{beameruserguide.pdf}}
  \end{flushleft}
\end{itemize}
\end{frame}




\end{document}
